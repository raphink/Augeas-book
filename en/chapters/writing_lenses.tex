\chapter{Writing Your Own Lenses}

\label{chap:writing_lenses} \index{Lenses!writing}

Augeas comes with a set of various lenses which cover most of the basic configuration files on a Unix machine. However, there are so many configuration file formats on Unix systems, that you are very likely to miss one at some point.

Augeas lenses are written in a ML language that is similar to OCaml. The language consists mostly of regexps and operators to combine them.

\section{A simple example}

Since Augeas lenses are mostly a combination of regular expressions that are often complex and fragile, it is safer to consider writing unit tests for each lens to ensure non-regression and confirm that all known cases are met by the lens. Our work will thus begin with the writing of a unit test, which will specify the way we will map the configuration entries to the Augeas tree. For extended information on unit tests, see the end of this chapter.

\subsection{Unit Test}

\textbf{Example of a simple key/value conffilem, step by step}

\subsection{Module}

\textbf{Example of a simple key/value conffile, step by step}

\section{Regular expressions}

The bidirectional nature of the Augeas language imposes strict conditions on the language\footnote{See chapter \ref{chap:bx} on page \pageref{bx}}. This makes complex regular expressions languages such as PCRE hard to implement. For this reason, Augeas only supports POSIX simple regular expressions.

\textbf{Give Examples}

\section{Special keywords}

The Augeas language provides a set of keywords to build lenses.

\subsection{key}

\subsection{label}

\subsection{store}

\subsection{value}

\subsection{seq}

\subsection{rec}

\subsection{square}

\section{Combination Operators}

Augeas lenses are put together by assembling regular expressions with combination operators.

\subsection{Concatenation Operator}

\subsection{Union Operator}

\section{Filters and Autoload}

Augeas lenses need to specify which files they apply to. If they didn't, Augeas would have no way to know which lens to apply to which files. Trying to guess would be a really bad idea. For example, consider a file whose only content is the following:

\begin{minted}[fontsize=\scriptsize]{bash}
# this is a comment
\end{minted}

Many lenses are able to parse this line, and will mostly likely map it the same way. However, once a lens has been chosen for the file, the rest of the configuration statements are likely to be very different from one lens to another, so you are almost sure that the lens you chose will be wrong.

Each lenses may have one and only one autoload statement, involving a lens and a filter, such as the following:

\begin{minted}[fontsize=\scriptsize]{augeas}
autoload xfm
let lns = ...
let filter = incl "/etc/foo.conf"
let xfm = transform lns filter
\end{minted}


\section{Typechecking lenses}

Augeas comes with a command line tool called \verb!augparse! which can be used to typecheck lenses, checking that they meet the conditions to be used as bidirectional transforms.

\subsection{Typechecking recursive lenses}

\section{Unit tests}

We have mentionned the importance of unit tests in the beginning of this chapter. It is worth repeting it: unit tests are essential to the stability of an Augeas lens. Unit tests need to be well written and kept up-to-date with new features and bug fixes to ensure that the lens continues to work with the files it was written for.

Augeas provides keywords to achieve unit tests in both the get and put directions.

\section{Using Generic Modules}

Augeas provides special modules to ease the writing of lenses.

\subsection{The Util module}

The Util (\verb!util.aug!) module provides definitions of comments, empty lines and other utilities.

\textbf{List functions} and give examples.

\subsection{The Sep module}

The Sep (\verb!sep.aug!) module provides definitions for separators. \textbf{List functions} and give examples.

\begin{quote}
\includegraphics{../images/info.png} \emph{\texttt{Sep.opt\_space} is a synonym for \texttt{Util.indent}. Both are strictly equivalent, but it is clearer to use the former as a separator and the latter as an indentation.}

\end{quote}
\subsection{The Rx module}

The Rx (\verb!rx.aug!) module provides definitions for usual regular expressions. \textbf{List functions} and give examples.

\subsection{The Build module}

The Build (\verb!build.aug!) module provides definitions for usual constructions of regular expression. \textbf{List functions} and give examples.

\subsection{The IniFile module}

INI files are quite standard even on Unix systems. However, there are many different implementations and variations. The Inifile (\verb!inifile.aug!) module provides definitions to ease the writing of lenses for specific INI files. It is used as a basis for lenses such as Php (\verb!php.aug!), MySQL (\verb!mysql.aug!) or Puppet (\verb!puppet.aug!). \textbf{List functions} and give examples.

\section{Using your lens}

Augeas uses a search path to find its lenses. By default, it will search for lenses in \verb!$prefix/share/augeas/lenses! and \nolinkurl{/$prefix/share/augeas/lenses/dist}, where \verb!$prefix! is the compilation prefix, usually \nolinkurl{/usr}.

The \verb!dist! subdirectory is reserved for stock lenses, while the top directory can be used to store your own lenses.

If you prefer to store your lenses in another place, or just wish to try a new lens without installing it in your system, you can override this search path in several ways.

\subsection{Ignoring the stock modules}

\index{augtool!options!--nostdinc}

In order to ignore the default search path for lenses, you can use the \verb!--nostdinc! flag in \verb!augtool!.

\subsection{Adding your own directory of lenses}

\index{augtool!options!--include} \index{Environment variables!\textsc{augeas\_lens\_lib}}

Directories containing additional lenses can be added to the search path by using the \verb!--include! option in \verb!augtool!, or the \verb!AUGEAS_LENS_LIB! environment variable:

\begin{minted}[fontsize=\scriptsize]{console}
$ augtool --include mylenses
\end{minted}


