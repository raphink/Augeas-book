\chapter{Bidirectional transformations}

\label{chap:bx}

Augeas uses files called lenses, written in a specific language for Augeas, which is similar to OCaml. Lenses are programs that are said to be bidirectional.

\section{The Need for Bidirectional Transformations}

Traditional programs take data as input and produce data as output, but cannot use the same code to go from the output data back to the input data. In other words, traditional programs are not bidirectional, they work only in one direction.

For example, the following transformation:

\consolecode@$ echo "I have food in my fridge" | sed -e "s/foo/bar/g"@
\consolecode@I have bard in my fridge@

transforms all \verb!foo! occurrence into \verb!bar! in the string, but it cannot go back from a \verb!bar! string to a \verb!foo! string.

The need to transform from one format and back is quite common, and has traditionally been addressed by writing two programs, one for each direction of the transformation.

In recent years, the Harmony Project\footnote{\url{http://www.seas.upenn.edu/~harmony/}} has been working on the mathematical conditions for programs to be bidirectional, or even bijective. They came up with a language called Boomerang, which implements their theory.

Other projects such as biXid\footnote{\url{http://arbre.is.s.u-tokyo.ac.jp/~hahosoya/papers/bixid.pdf}} or XSugar\footnote{\url{http://www.brics.dk/xsugar/}} have also been working on this same goal concurrently.

XSugar provides a way to transform between XML and non-XML data models, while biXid allows transformations between two XML data models.

\section{A Bit of Theory}

\textbf{This might be a part for Francis}
\textbf{Also see with Nate if you can use/adapt his schemas}

\section{Bidirectional transformations in Augeas}


