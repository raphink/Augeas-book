\chapter{Using Augeas in Puppet}

\chquote{
Let it be observed, that slovenliness is no part of religion; \\
that neither this nor any text of Scripture,\\
condemns neatness of apparel.\\
Certainly this is a duty, not a sin,\\
``Cleanliness is, indeed, next to godliness.''\\
}{-- John Wesley, Sermon xciii, `on Dress'}

\index{Puppet|(}
Because Augeas is a configuration API, it fits right into tools that are made for configuration management. One of the most widely used of these tools in the open-source world is Puppet, and Augeas has been available as a native type in Puppet since version 0.24.7.

\index{API!bindings!Ruby}
Since Puppet is written in Ruby, the Augeas Puppet type makes use of the Ruby bindings for Augeas.

\section{The Augeas type}

Puppet provides a native Augeas type since version 0.24.7.

\index{Commands!set} The Augeas type in Puppet takes a list of commands labeled ``changes''. The example of listing \ref{lst:augtool_set_single}\footnoteref{lst:augtool_set_single} then becomes:

\begin{puppet-augeas}[]
augeas { "hosts_alice":
   changes => [
      "set /files/etc/hosts/1/canonical alice",
   ],
}
\end{puppet-augeas}

The \verb!changes! attribute is an array of Augeas commands, similar to what you would pass to \verb!augtool!.

\begin{quote}
\info{It is recommended to use \texttt{augtool} to prepare and test the commands before you use them in Puppet.}

\end{quote}
\index{Commands!save} Each call to the Augeas type starts a new Augeas session. The \verb!save! call is ran automatically at the end of each session.

\section{Setting a context}

\section{Proper quoting}

While quoting in \verb!augtool! is strict, quoting in Puppet can be tricky.

\textbf{TODO}: explain this

\section{Puppet and idempotence}

Idempotence is very important in configuration management tools such as Puppet. The Augeas type provides a \verb!onlyif! statement to make it easy to ensure that Augeas is only called when necessary.

\begin{puppet-augeas}[]
augeas { "hosts_alice":
   context => "/files/etc/hosts/1",
   changes => [
      "set canonical alice",
   ],
   onlyif => "match canonical[. = 'alice'] size == 0",
}
\end{puppet-augeas}

\begin{quote}
\info{For proper idempotence, this statement has to be coupled with the methods described earlier\footnoteref{sec:ensuring_idempotence}.}
\end{quote}

\index{Puppet|)}
