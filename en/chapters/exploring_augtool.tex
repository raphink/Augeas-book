\chapter{Exploring augtool}

\label{chap:augtool}
\index{augtool}
\index{augtool!commands|see{Commands}}

\chquote{King Augeas’ fleecy flocks, good Sir,\\
feed not all of one pasture nor all upon one spot,\\
but some of them be tended along Heilisson,\\
others beside divine Alpheüs’ sacred stream,\\
others again by the fair vineyards of Buprasium,\\
and yet others, look you, hereabout;\\
and each flock hath his several fold builded.}{Idyll 25.7}{Theocritus}

Augeas is primarily a C library with bindings but it also provides a command-line tool called \verb!augtool!, which we will be using in the following examples. In chapter~\ref{chap:api}, we will see how to use the C~API and bindings directly.


\section{Parsing your System Configuration Files}

The first thing you might want to do is to explore how Augeas sees your system configuration files. Fire up \verb!augtool!:

\consolecode/$ augtool/

This will give you an interactive shell which passes commands to Augeas.

To see which commands your version of \texttt{augtool} supports, simply type \texttt{help}:

\nopagebreak
\begin{augtoolsh}[]
augtool> help

Admin commands:
  help       - print help
  load       - (re)load files under /files
  quit       - exit the program
  retrieve   - transform tree into text
  save       - save all pending changes
  store      - parse text into tree
  transform  - add a file transform
\end{augtoolsh}
\begin{augtoolsh}[]
Read commands:
  dump-xml   - print a subtree as XML
  get        - get the value of a node
  label      - get the label of a node
  ls         - list children of a node
  match      - print matches for a path expression
  print      - print a subtree
  span       - print position in input file corresponding to tree
\end{augtoolsh}
\begin{augtoolsh}[]
Write commands:
  clear      - clear the value of a node
  clearm     - clear the value of multiple nodes
  ins        - insert new node
  insert     - insert new node (alias of 'ins')
  mv         - move a subtree
  move       - move a subtree (alias of 'mv')
  rename     - rename a subtree label
  rm         - delete nodes and subtrees
  set        - set the value of a node
  setm       - set the value of multiple nodes
  touch      - create a new node
\end{augtoolsh}
\begin{augtoolsh}[]
Path expression commands:
  defnode    - set a variable, possibly creating a new node
  defvar     - set a variable

Type 'help <command>' for more information on a command
\end{augtoolsh}


Augeas transforms your configuration files into a tree with two nodes at its root: \texttt{/augeas} and \texttt{/files}.
 The \texttt{/augeas} node contains metadata, which we will be looking at in chapter~\ref{chap:metadata},
 while \texttt{/files} contains the representation of the files Augeas was able to parse.
 You can inspect these two nodes by typing \verb!ls /!:

\index{Commands!ls}

\begin{augtoolsh}[]
augtool> ls /
augeas/ = (none)
files/ = (none)
\end{augtoolsh}

What does this mean? We see the two nodes at the top of the Augeas tree, and that neither of them has a value. In the Augeas tree, each node can have children and a value associated with it.

\verb!ls! is an \verb!augtool! command which lists the children of the given node and gives their value (if any).

You can see which files (or directories containing files) were successfully parsed by Augeas in \texttt{/etc} by typing \verb!ls /files/etc!:

\begin{augtoolsh}[]
augtool> ls /files/etc
nsswitch.conf/ = (none)
odbc.ini = (none)
passwd/ = (none)
ntp.conf/ = (none)
services/ = (none)
sysctl.conf/ = (none)
shells/ = (none)
samba/ = (none)
securetty/ = (none)
crypttab/ = (none)
...
\end{augtoolsh}

\index{Commands!print}

Let's inspect the contents of the first line of \texttt{/etc/fstab} in the Augeas tree. We can use the \verb!print! command to inspect nodes and their values recursively:

\begin{augtoolsh}[]
augtool> print /files/etc/fstab/1
/files/etc/fstab/1
/files/etc/fstab/1/spec = "proc"
/files/etc/fstab/1/file = "/proc"
/files/etc/fstab/1/vfstype = "proc"
/files/etc/fstab/1/opt[1] = "nodev"
/files/etc/fstab/1/opt[2] = "noexec"
/files/etc/fstab/1/opt[3] = "nosuid"
/files/etc/fstab/1/dump = "0"
/files/etc/fstab/1/passno = "0"
\end{augtoolsh}

Each of the child nodes beneath the \verb!1! node refers to a part of a single line in the \texttt{/etc/fstab} file. The filesystem options are further split into separate nodes under the \verb!opt! node so they can be managed individually.

What if we only wanted to find the \verb!opt! nodes of this first line? The \verb!match! command lets us find the nodes matching an expression:

\index{Commands!match}

\begin{augtoolsh}[]
augtool> match /files/etc/fstab/1/opt
/files/etc/fstab/1/opt[1] = nodev
/files/etc/fstab/1/opt[2] = noexec
/files/etc/fstab/1/opt[3] = nosuid
\end{augtoolsh}

\begin{quote}
  \info{In \texttt{augtool}, \texttt{match} returns values as well as found paths. In the API however, only paths are returned, without the associated values.}
\end{quote}

Now, you might want to get the value of the single node matching an expression, and make sure that this node is unique. For example, if we want the value of the first \verb!opt! node of this first line, we could use the \verb!get! command:

\begin{augtoolsh}[]
augtool> get /files/etc/fstab/1/opt[1]
/files/etc/fstab/1/opt[1] = nodev
\end{augtoolsh}

\index{Commands!quit}

To leave the \verb!augtool! session, you can type \verb!quit! or \verb!^D!:

\augtoolshcode/augtool> quit/


\section{Using a Fakeroot}

It is often useful to play with \verb!augtool! when you want to understand the Augeas tree or try XPath expressions. However, you likely don't want to play with the files in your \texttt{/etc} directory and take the risk of ruining your system. Augeas lets you set a fake root so that the files parsed and modified by Augeas are taken from this root instead of the \texttt{/} directory of your system.

\index{augtool!options!--root} \index{Environment variables!\textsc{augeas\_root}}

In \verb!augtool! you can set this fakeroot by using the \verb!--root! (or \verb!-r!) option:

\begin{console}[]
$ mkdir -p myroot/etc
$ rsync -av /etc/ myroot/etc
$ augtool -r myroot
\end{console}

In general, you can also set the location of this fakeroot with the \verb!AUGEAS_ROOT! environment variable:

\begin{console}[]
$ export AUGEAS_ROOT="$(pwd)/myroot"
$ augtool
\end{console}

This option can also let you modify files inside a chroot for example.

\section{Modifying Files}

We have seen already how Augeas lets you parse your configuration files in a unified way. The Augeas tree is not only a parsing facility as Augeas exposes commands to modify the tree and save the changes to the original files.

The \verb!--root! option will be useful for us here, in order to modify the files without affecting the system. We will also use the \verb!--backup! option in \verb!augtool! so that the original files are preserved with a \verb!.augsave! extension.

\index{augtool!options!--backup} \index{augtool!options!--root} \index{Commands!rm} \index{Commands!quit} \index{Commands!save} \index{Environment variables!\textsc{augeas\_root}}

\begin{listing}
  \consolecode[linenos]|$ augtool --backup --root myroot|
  \myinputminted[linenos,firstnumber=2]{augtool-shell}{rm_fstab_opt.augtoolshell}
  \consolecode[linenos,firstnumber=15]|$ diff -u myroot/etc/fstab myroot/etc/fstab.augsave|
  \myinputminted[linenos,firstnumber=16]{diff}{fstab_opt.diff}
  \caption{Removing an option in fstab}
  \label{lst:rm_fstab_opt}
\end{listing}


In listing \refandpage{lst:rm_fstab_opt} we change the filesystem options specified on the first line of \texttt{/etc/fstab} by removing the third \verb!opt! node. The \verb!rm! command on line 2 removes only the \verb!opt! node we specified. Line 3 tells us that the \verb!rm! command removed only one node, the \texttt{/files/etc/fstab/1/opt[3]} node. Lines 4 through 11 show us the \texttt{/files/etc/fstab/1} tree without the removed node.

On line 12, we call the \verb!save! command. This command tells Augeas to save the tree back to the configuration files. Augeas inspects the files and tries to apply the new tree to them. In our case, the \verb!save! command was successful as line 13 tells us, and one file was modified, which is what we expected. We can then quit the \verb!augtool! session by typing \verb!quit! on line 14.

We use the \verb!diff -u! command on line 15 to inspect the changes made by Augeas to the file. As expected, only the first line that is not empty or a comment was modified. Lines 22 and 23 in the listing show us the differences between the old and new lines. We can see that only the third option has been removed, and that the spaces have been strictly preserved. The rest of the file was left untouched.

\section{Preserving existing files}

\index{augtool!options!--backup} \index{augtool!options!--new} \index{Metadata!\slash{}augeas\slash{}save}

Augeas offers two options to preserve the existing files when saving the tree. In \verb!augtool!, these options can be triggered with the following flags:

\begin{itemize}
\item
  \verb!--backup! will save the original file with the extension .augsave and write the new file under the original file name;
\item
  \verb!--new! will save the modified file with a \verb!.augnew! extension and leave the original file untouched.
\end{itemize}

These options actually modify the value of the \texttt{/augeas/save} node in the Augeas tree.\footnoteref{sec:save_node}

\section{Locating nodes in files}

\label{sec:locating_nodes} \index{augtool!options!--span} \index{Flags!\textsc{aug\_enable\_span}}

The span metadata were added in Augeas 0.8.0. For performance reasons, they are not activated by default. This functionality can be activated by setting the \verb!AUG_ENABLE_SPAN! flag or using the \verb!--span! flag in \verb!augtool!.

You can see if the \verb!span! functionality is activated in the current session by looking at the \texttt{/augeas/span} node:\footnoteref{sec:span_node}

\index{Metadata!\slash{}augeas\slash{}span}

\begin{augtoolsh}[]
augtool> get /augeas/span
/augeas/span = enable
\end{augtoolsh}

The data are then available via the \verb!span! command in \verb!augtool!.

\index{Commands!span}

\begin{listing}
  \consolecode[linenos]|$ augtool --span|
  \myinputminted[linenos,firstnumber=2]{augtool-shell}{span_ntp_driftfile.augtoolshell}
  \begin{console}[linenos,firstnumber=7]
$ head -c100 /etc/ntp.conf  | tail -c+67

driftfile /var/lib/ntp/ntp.drift
  \end{console}
  \caption{Getting the position of a node with span}
  \label{lst:span_ntp}
\end{listing}

Line 5 in listing \refandpage{lst:span_ntp} indicates that:

\begin{itemize}
\item
  The \verb!driftfile! label was found in the file between positions 67 and 76. This also means that \verb!driftfile! is a dynamic key, not a static label;\footnote{It was declared with the \verb!key! keyword, not with \verb!label!. See chapter \refandpage{chap:writing_lenses}.}
\item
  The value of the \verb!driftfile! node was found between positions 77 and 99 in the file;
\item
  The whole span of the node is between positions 67 and 100 in the file. The span is one character further than the value, since the \verb!\n! character is considered part of the lens matching the node, but is excluded from the value.
\end{itemize}

We verify on line 9 that the data located between positions 67 and 100 in the file correspond to the \verb!driftfile! key and the value returned by the \verb!get! command on line 3.


\section{Scripting with augtool}

\index{augtool!scripting}

In addition to running as an interactive shell, \verb!augtool! can take commands from the command line or \verb!stdin!:

\index{Commands!ls} \index{augtool!piping}

\consolecode[linenos]|$ augtool ls /files|
\augtoolcode[linenos,firstnumber=2]|etc/ = (none)|
\consolecode[linenos,firstnumber=3]@$ echo "ls /files/" | augtool@
\augtoolcode[linenos,firstnumber=4]|etc/ = (none)|

This allows to write shell scripts that send commands to \verb!augtool!.

\index{Commands!set} \index{Commands!save} \index{augtool!piping}

\begin{listing}
  \myinputminted[linenos]{bash}{augtool_wrap.sh}
  \caption{Piping commands to augtool in a bash script}
  \label{lst:augtool_wrap}
\end{listing}

\index{augtool!options!--autosave}

\begin{quote}
\info{The \texttt{--autosave} option in \texttt{augtool} allows you to ommit the \texttt{save} command.}
\end{quote}

Listing \refandpage{lst:augtool_wrap} shows an example of a bash script wrapping around \verb!augtool!. Lines 2 through 5 define the wrapping function \verb!do_augtool! which is then called on line 7. Commands are separated with \verb!\n! so they get passed line by line to \verb!augtool! through \verb!echo -e!.


\subsection{Using augtool as an interpreter}

\verb!augtool! can also take commands from a file:

\index{augtool!options!--file} \index{Commands!ls}

\begin{listing}
  \consolecode|$ cat commands.augtool|
  \augtoolcode|ls "/files"|
  \consolecode|$ augtool --file commands.augtool|
  \augtoolcode|etc/ = (none)|
  \caption{\texttt{augtool} takes a command file as argument}
  \label{lst:augtool_file_arg}
\end{listing}

This allows to use \verb!augtool! as a script interpreter in a shebang and write self-executable \verb!augtool! scripts (using the \verb!-f! short version of the option):

\begin{listing}
  \consolecode|$ cat commands.augtool|
  \augtoolcode|#!/usr/bin/augtool -f|
  \augtoolcode|ls "/files"|
  \consolecode|$ chmod +x commands.augtool|
  \consolecode|$ ./commands.augtool|
  \augtoolshcode|etc/ = (none)|
  \caption{Using \texttt{augtool} as an interpreter}
  \label{lst:augtool_interpreter}
\end{listing}

\subsection{Dropping into an interactive session}

When \verb!augtool! takes commands from the command line, \verb!stdin! or a file, it doesn't start an interactive session. If you wish to pass commands to \verb!augtool! for preprocessing and run an interactive command afterwards, you can use the \verb!--interactive! flag:

\index{augtool!piping} \index{Commands!get} \index{augtool!options!--interactive}

\begin{listing}
  \consolecode@$ echo "set /files/etc/hosts/1/canonical alice" | augtool --interactive@
  \augtoolshcode|augtool> get /files/etc/hosts/1/canonical|
  \augtoolshcode|/files/etc/hosts/1/canonical = alice|
  \caption{Setting a single value in \texttt{augtool}}
  \label{lst:augtool_set_single}
\end{listing}

\begin{quote}
\info{The \texttt{--interactive} option only works for \texttt{stdin} and file input.}

\end{quote}
This option also allows you to make scripts that set up an environment and drop you in an interactive shell:

\index{augtool!options!--file} \index{Commands!get} \index{Commands!set} \index{Commands!quit}

\consolecode|$ cat shell.augtool|
\augtoolcode|#!/usr/bin/augtool -if|
\augtoolcode|set /files/etc/hosts/1/canonical alice|
\consolecode|$ chmod +x shell.augtool|
\consolecode|$ ./shell.augtool|
\begin{augtoolsh}[]
augtool> get /files/etc/hosts/1/canonical
/files/etc/hosts/1/canonical = alice
augtool> quit
\end{augtoolsh}

\begin{quote}
\info{Only concatenated short options can be used in shebangs, hence the use of \texttt{-if}.}
\end{quote}
