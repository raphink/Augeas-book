% Use koma?
\documentclass[a4paper,12pt]{scrbook}
%\documentclass{book}
\usepackage[english]{babel} %language selection
\selectlanguage{english}
\pagenumbering{arabic}

\usepackage{amsmath}
\usepackage[mathletters]{ucs}
\usepackage[utf8x]{inputenc}
\usepackage{fancyvrb}
\usepackage{graphicx}
\usepackage[breaklinks=true,unicode=true]{hyperref}
\hypersetup{colorlinks, 
           citecolor=black,
           filecolor=black,
           linkcolor=black,
           urlcolor=black,
           bookmarksopen=true,
           pdftex}
\usepackage{url}
\usepackage[activate={true,nocompatibility}]{microtype}

\hfuzz = .6pt % avoid black boxes

\setlength{\parindent}{0pt}
\setlength{\parskip}{6pt plus 2pt minus 1pt}
\setcounter{secnumdepth}{0}
\VerbatimFootnotes % allows verbatim text in footnotes

% Make indexes
\usepackage{makeidx}
\makeindex


%% minted conf

% Use minted for syntax highlighting
\usepackage{minted}
% See pygmentize -L styles for the list of styles
%\usemintedstyle{pastie}

% Make nice looking line numbers
\renewcommand{\theFancyVerbLine}{\sffamily
\textcolor[rgb]{0.5,0.5,1.0}{\scriptsize
\oldstylenums{\arabic{FancyVerbLine}}}}

% Code bg color
\definecolor{bg}{rgb}{0.95,0.95,0.95}


% Define macros
\newcommand{\seeref}[1]{See~\emph{\nameref{#1}} on page~\pageref{#1}}
\newcommand{\footnoteref}[1]{\footnote{\seeref{#1}.}}


\title{Augeas: A configuration API}
\author{Raphaël Pinson}

\begin{document}
\maketitle

\cleardoublepage
Augeas: A configuration API

Copyright \copyright~2011 Raphaël Pinson

Permission is granted to copy, distribute and/or modify this document under the terms of the GNU Free Documentation License, Version 1.3 or any later version published by the Free Software Foundation; with the Invariant Sections being just "GNU Manifesto", with no Front-Cover Texts, and with no Back-Cover Texts.  A copy of the license is included in the section entitled "GNU Free Documentation License".

This book would not be what it is without the help of many.
In particular, I wish to thank David Lutterkort for writing Augeas and providing great insights about Augeas internals.
Doing proper typesetting with \LaTeX can be tricky, and Damien Wyart deserves a big thanks for providing his expertise on the subject.



\tableofcontents

\cleardoublepage
\phantomsection
\addcontentsline{toc}{chapter}{Introduction}
\chapter*{Introduction}

\chquote{Augeias was king of Elis,\\
in some accounts a son of Helios,\\
in others of Poseidon,\\
and in still others of Phorbas.}{The Library 2.88}{Apollodorus}

In the world of Unix systems, there is no standard way to store configuration. Countless formats can be found, from simple shell variable files to complex, specific, multi-level formats, making the infamous /etc directory a sort of digital Augean stable.

\begin{table}
\begin{center}
  \begin{tabular}{l|l}
    \rowcolor{gray!50}
    \textsc{Configuration file} & \textsc{Format} \\
    \hline \hline
    \verb!/etc/default/*! & shell variables \\
    \hline
    \verb!/etc/fstab! & \multirow{2}{*}{fstab format} \\
    \verb!/etc/mtab! &  \\
    \hline
    \verb!/etc/hosts! & hosts format \\
    \hline
    \verb!/etc/passwd! & \multirow{2}{*}{passwd format} \\
    \verb!/etc/shadow! & \\
    \hline
    \verb!php.ini! & \multirow{4}{*}{\textsc{INI} file} \\
    \verb!my.cnf! & \\
    \verb!gdm.conf! &  \\
    \verb!puppet.conf! &  \\
    \hline
    \verb!ntp.conf! & \textsc{NTP} format \\
  \end{tabular}
  \legend{Some common configuration files and their format}
\end{center}
\end{table}


Augeas provides a way to cleanly and safely manage these configuration files through a unified API.

\section{Configuration Data Editing Approaches}

While system administrators are well aware of the heterogenous state of the configuration data on Unix systems, these configurations have to be edited automatically in many situations.

There are three main approaches to the issue of automating configuration data editing on Unix systems.

\subsection{Keyhole Approaches}

Most programming languages provide modules to edit at least the most common formats, but a lot of system administrators and developers manipulate these files using string editing tools such as \texttt{sed}, \texttt{awk} or \texttt{cut}, or even write scripts dedicated to a specific parsing job. In the majority of these cases, the results are not guaranteed, and you are likely to ruin the configuration files if your parsing expressions are wrong or the file layout changes between different versions of the program.

Configuration management tools such as Cfengine\footnote{\url{http://www.cfengine.com}} provide tools to achieve keyhole approaches, but the problems are similar to using string editing tools: you have no guarantee that the result will be a valid configuration file, and you have to write the regexps yourself.

\todo{Provide examples of using sed or awk}

Augeas is particularly useful to ease and secure this kind of approach.

\subsection{Greenfield approaches}

When you are the main system administrator of a machine and you wish to control all the parameters of the machine, you may want to provide the configuration files entirely. In this case, it is common to set up a repository of configuration files, or a database, which will contain the whole configuration as will be deployed to the machines. This is often the best solution but is not always possible.

\todo{Maybe provide an example such organization or schema}

\subsection{Templating}

If you wish to control whole configuration files but you need a fine-grained mechanism to generate these files, templating is probably the best approach. There are lots of options to achieve this. Puppet\footnote{\url{http://www.puppetlabs.com}}, for example, provides ERB templates that let you easily generate configuration files from exported variables.

\todo{Given an example of an ERB template for Puppet for example}

\section{A Unified Configuration API}

In many cases, system administrators, developers and users want to change a single value in their configuration without affecting the rest of it. This is often achieved using the keyhole approach, which as we have seen is not very reliable. A better approach would be to have a unified configuration API that lets you modify configurations in a simple and reliable way, ensuring that the modified files are valid configuration files. This is the goal of Augeas.

\section{What Augeas is not}

A principle on Unix systems ensures the stability and simplicity of the system tools: each tool attempts to do one thing, and to do it well. Augeas is no exception to this rule, so Augeas is as much defined by the things it does not try to accomplish as by its goals.

Before we dive into what Augeas can do for you, it is important to note the following points.

\subsection{Not an abstraction layer}

Augeas does not attempt to provide an abstraction layer from the native configuration format. The organization of the Augeas tree mirrors closely that of the configuration files it represents.

As an example, if the \nolinkurl{/etc/foo.conf} configuration file contains an include statement such as the following:

\begin{bash}[]
#include /etc/foo.d/*
\end{bash}

Augeas will not attempt to parse the contents of the files in \nolinkurl{/etc/foo.d/*} and add them to the \nolinkurl{/etc/foo.conf} tree. Instead, it will provide a tree like the following:

\begin{augtoolsh}[]
/files/etc/foo.conf
/files/etc/foo.conf/#include = /etc/foo.d/*
\end{augtoolsh}

\verb!#include! is just a parameter of the \nolinkurl{/etc/foo.conf} configuration file and \nolinkurl{/etc/foo.d/*} is the value of this parameter. The contents of \nolinkurl{/etc/foo.d/*} will probably appear in the tree if the lens is able to parse them, but in no way will Augeas make a logical link between \nolinkurl{/etc/foo.conf} and \nolinkurl{/etc/foo.d/*}.

Other software provide this kind of abstraction layer. This is the case of \texttt{Config::Model}, which can use Augeas as a backend, and is able to understand the logic of a configuration files, such as include statements, or the link between several statements in a configuration file.

Another consequence of this non-goal is that the statements in the Augeas tree will appear in the same order as they do in the configuration file. In some cases, it is technically possible to write Augeas lenses that invert parameters or otherwise modify the logic of the configuration statements. Doing this is not recommended, as the Augeas tree should stay as close as possible in its logic to the configuration files it is representing in order to provide maximum flexibility.

\subsection{Not a cross-platform abstraction layer}

For a similar reason, Augeas does not attempt to be a cross-platform abstraction layer. When Augeas finds Apache configuration files in \nolinkurl{/etc/httpd/httpd.conf} on some operating systems and in \nolinkurl{/etc/apache2/apache2.conf} in others, these files will be represented in the tree as \nolinkurl{/files/etc/httpd/httpd.conf} and \nolinkurl{/files/etc/apache2/apache2.conf} respectively.

Similarly, some operating systems provide their network configuration in \nolinkurl{/etc/sysconfig/network} while others use \nolinkurl{/etc/network/interfaces}. Augeas will represent these two files in different parts of the tree, and the tree will mirror the way each of these files is organized, without attempting to provide a unified way to configure network interfaces across these operating systems.

Other projects such as netcf\footnote{\url{https://fedorahosted.org/netcf}}, based on Augeas, provide a cross-platform abstraction layer to manage network interfaces regardless of the operating system, but it is not Augeas' goal.

\subsection{No remote management support}

When you are dealing with a whole fleet of servers and wish to set a parameter for each of them, it is useful to use a tool that has a network protocol for remote management. Augeas does not attempt to be that tool, and the Augeas API is designed to be a local API.

Remote access to the Augeas API are meant to be added on top of it, not in it.

Puppet is an example of configuration management tool which supports Augeas as a native type and provides remote management functionality\footnote{\seeref{chap:puppet}.}. Mcollective with the \texttt{augeasquery} agent also provides a remote access to Augeas\footnote{\seeref{sec:mcollective}}.

\subsection{Very little modelling}

The goal of Augeas is not to understand or otherwise interpret configuration files. As stated before, Augeas does not attempt to provide an abstraction layer, but it provides a light modelling, although very close to the organization of the configuration files.

For example, an \nolinkurl{/etc/hosts} line like the following:

\begin{bash}[]
192.168.0.10    aslan   # Added by NetworkManager
\end{bash}

will be represented by the following tree:

\begin{augtoolsh}[]
/files/etc/hosts
/files/etc/hosts/1
/files/etc/hosts/1/ipaddr = "192.168.0.10"
/files/etc/hosts/1/canonical = "aslan"
/files/etc/hosts/1/#comment = "Added by NetworkManager"
\end{augtoolsh}

The order of the statements is strictly kept; Augeas does not interpret the configuration files per se, but it labels each of the fields on the line to ease access to individual configuration items.


\subsection{Conventions}

This book uses the following conventions:
\begin{itemize}
  \item
    Filesystem paths, Augeas calls and Unix commands are typeset in a monospace font;
  \item
    When lines are too long in an output, an antislash (\verb!\!) is added and the rest of the line is reported to the next line with an indentation.
\end{itemize}


\todo{Refine conventions}


\subsection{How to read this book}

The first chapter of this book is about installing Augeas. You can skip this chapter if you already have Augeas installed on your machine.

This book is intended primarily for developers and systems administrators.
 Whether you are one or the other, you should read chapter~\ref{chap:augtool} to understand what Augeas is about if you are not familiar with it yet.

Chapter~\ref{chap:bx} introduces the reader to the theory of Biridectional Transformations, on which Augeas lenses are built. You can safely skip this chapter if you do not intend to write lenses, but it is strongly recommended to read it otherwise.

Path Expressions, presented in chapter~\ref{chap:pathx}, is an essential part of using Augeas. This chapter will assume that you know the basics about XPath expressions\footnote{\url{http://www.w3schools.com/xpath}}.

Developers will most likely be interested in the C API and bindings presented in chapter~\ref{chap:api}.

The Augeas metadata which will be talked about in chapter~\ref{chap:metadata} are useful for all kinds of tree manipulation, and all users should probably go through it.

Chapter~\ref{chap:puppet} is specifically about using Augeas in Puppet and Mcollective. It will mostly be interesting to systems administrators using these tools.

Should you need to write your own lenses, chapter~\ref{chap:writing_lenses} will help you reach this goal.

Finally, chapter~\ref{chap:troubleshooting} will teach you how to troubleshoot Augeas, which will be useful to all users.


\subsection{Prerequisites}

This book will assume that you are familiar with Unix configuration files in general.

Chapter~\ref{chap:pathx} will use XPath expressions. Knowing standard XPath implementations will make this chapter easier to understand.

Augeas lenses are written in a language similar to OCaml. Knowing this language, or another ML language, will greatly help you to understand how to write lenses.






\chapter{Exploring augtool}

\index{augtool}
\index{augtool!commands|see{Commands}}

\begin{verse}
  \begin{flushright}
    \begin{scriptsize}
``King Augeas’ fleecy flocks, good Sir, \\
feed not all of one pasture nor all upon one spot, \\
but some of them be tended along Heilisson, \\
others beside divine Alpheüs’ sacred stream, \\
others again by the fair vineyards of Buprasium, \\
and yet others, look you, hereabout; \\
and each flock hath his several fold builded.'' \\
\bigskip
\tiny{-- Theocritus, Idyll 25.7}
    \end{scriptsize}
  \end{flushright}
\end{verse}
\bigskip

While Augeas is a C library with bindings, it also provides a command-line tool called \verb!augtool!, which we will be using in the following examples. In chapter~\ref{chap:api}, we will see how to use the API and bindings directly.


\section{Parsing your System Configuration Files}

The first thing you might want to do is to see how Augeas sees your system configuration files. Fire up \verb!augtool!:

\begin{minted}[fontsize=\scriptsize]{console}
$ augtool
\end{minted}

This will give you an interactive shell which passes commands to Augeas. Augeas transforms your configuration files into a tree, which has two nodes at its root: \nolinkurl{/augeas} and \nolinkurl{/files}. The \nolinkurl{/augeas} node contains metadata, which we will be looking at later on, while \nolinkurl{/files} contains the representation of the files Augeas was able to parse. You can see these two nodes by typing \verb!ls /!:

\index{Commands!ls}

\begin{minted}[fontsize=\scriptsize]{augtool-shell}
augtool> ls /
augeas/ = (none)
files/ = (none)
\end{minted}

What does that mean? We see the two nodes at the top of the Augeas tree, and we see that neither of them has a value. In the Augeas tree, each node can have children and a value associated with it.

\verb!ls! is an \verb!augtool! command which lists the children of the given node and gives their value (if any).

You can see which files (or directories containing files) were successfully parsed by Augeas in \nolinkurl{/etc} by typing \verb!ls /files/etc!:

\begin{minted}[fontsize=\scriptsize]{augtool-shell}
augtool> ls /files/etc
nsswitch.conf/ = (none)
odbc.ini = (none)
passwd/ = (none)
ntp.conf/ = (none)
services/ = (none)
sysctl.conf/ = (none)
shells/ = (none)
samba/ = (none)
securetty/ = (none)
crypttab/ = (none)
...
\end{minted}

\index{Commands!print}

Let's inspect the contents of the first line of \nolinkurl{/etc/fstab} in the Augeas tree. We can use the \verb!print! command to inspect nodes and their values recursively:

\begin{minted}[fontsize=\scriptsize]{augtool-shell}
augtool> print /files/etc/fstab/1
/files/etc/fstab/1
/files/etc/fstab/1/spec = "proc"
/files/etc/fstab/1/file = "/proc"
/files/etc/fstab/1/vfstype = "proc"
/files/etc/fstab/1/opt[1] = "nodev"
/files/etc/fstab/1/opt[2] = "noexec"
/files/etc/fstab/1/opt[3] = "nosuid"
/files/etc/fstab/1/dump = "0"
/files/etc/fstab/1/passno = "0"
\end{minted}

Each of the child nodes beneath the \verb!1! node refers to a part of a single line in the \nolinkurl{/etc/fstab} file. The filesystem options are further split into separate nodes under the \verb!opt! node so they can be managed individually.

What if we only wanted to find the \verb!opt! nodes of this first line? The \verb!match! command lets us find the nodes matching an expression:

\index{Commands!match}

\begin{minted}[fontsize=\scriptsize]{augtool-shell}
augtool> match /files/etc/fstab/1/opt
/files/etc/fstab/1/opt[1] = nodev
/files/etc/fstab/1/opt[2] = noexec
/files/etc/fstab/1/opt[3] = nosuid
\end{minted}

Now, we might want to get the value of the single node matching an expression, and make sure that this node is unique. For example, if we want the value of the first \verb!opt! node of this first line, we could use the \verb!get! command:

\begin{minted}[fontsize=\scriptsize]{augtool-shell}
augtool> get /files/etc/fstab/1/opt[1]
/files/etc/fstab/1/opt[1] = nodev
\end{minted}

\index{Commands!quit}

To leave the \verb!augtool! session, you can type \verb!quit! or \verb!^D!:

\begin{minted}[fontsize=\scriptsize]{augtool-shell}
augtool> quit
\end{minted}


\section{Using a Fakeroot}

It is often useful to play with \verb!augtool! when you want to understand the Augeas tree or try XPath expressions. However, you likely don't want to play with the files in your \nolinkurl{/etc} directory and take the risk of ruining your system. Augeas lets you set a fakeroot so that the files parsed and modified by Augeas are taken from this root instead of the \nolinkurl{/} directory of your system.

\index{augtool!options!--root} \index{Environment variables!\textsc{augeas\_root}}

In \verb!augtool! you can set this fakeroot by using the \verb!--root! option:

\begin{minted}[fontsize=\scriptsize]{console}
$ mkdir -p myroot/etc
$ rsync -av /etc/ myroot/etc
$ augtool -r myroot
\end{minted}

In general, you can also set the location of this fakeroot with the \verb!AUGEAS_ROOT! environment variable:

\begin{minted}[fontsize=\scriptsize]{console}
$ export AUGEAS_ROOT="$(pwd)/myroot"
$ augtool
\end{minted}

This option can also let you modify files inside a chroot for example.

\section{Modifying Files}

We have seen already how Augeas lets you parse your configuration files in a unified way. The Augeas tree is not only a parsing facility as Augeas exposes commands to modify the tree and save the changes to the original files.

The fakeroot option will be useful for us here, in order to modify the files without affecting the system. We will also use the \verb!--backup! option in \verb!augtool! so that the original files are preserved with a \verb!.augsave! extension.

\index{augtool!options!--backup} \index{augtool!options!--root} \index{Commands!rm} \index{Commands!quit} \index{Commands!save} \index{Environment variables!\textsc{augeas\_root}}

\begin{listing}
  \begin{minted}[fontsize=\scriptsize]{console}
$ augtool --backup --root myroot
  \end{minted}
  \begin{minted}[fontsize=\scriptsize]{augtool-shell}
augtool> rm /files/etc/fstab/1/opt[3]
rm : /files/etc/fstab/1/opt[3] 1
augtool> print /files/etc/fstab/1
/files/etc/fstab/1 /files/etc/fstab/1/spec = "proc"
/files/etc/fstab/1/file = "/proc"
/files/etc/fstab/1/vfstype = "proc"
/files/etc/fstab/1/opt[1] = "nodev"
/files/etc/fstab/1/opt[2] = "noexec"
/files/etc/fstab/1/dump = "0"
/files/etc/fstab/1/passno = "0"
augtool> save
Saved 1 file(s)
augtool> quit
  \end{minted}
  \begin{minted}[fontsize=\scriptsize]{console}
$ diff -u myroot/etc/fstab myroot/etc/fstab.augsave
  \end{minted}
  \inputminted[fontsize=\scriptsize]{diff}{../listings/fstab_opt.diff}
  \caption{Removing an option in fstab}
  \label{lst:rm_fstab_opt}
\end{listing}


In listing \ref{lst:rm_fstab_opt}, we change the filesystem options specified on the first line of \nolinkurl{/etc/fstab} by removing the third \verb!opt! node. The \verb!rm! command removes only the \verb!opt! node we specified, and the saved file has only this option removed. The rest of the file and even the rest of this line was left untouched, preserving the original formatting and layout.

\section{Preserving existing files}

\index{augtool!options!--backup} \index{augtool!options!--new} \index{Metadata!\slash{}augeas\slash{}save}

Augeas offers two options to preserve the existing files when saving the tree. In \verb!augtool!, these options can be triggered with the following flags:

\begin{itemize}
\item
  ---backup will save the original file with the extension .augsave and write the new file under the original file name;
\item
  ---new will save the modified file with a .augnew extension and leave the original file untouched.
\end{itemize}
These options actually modify the value of the \nolinkurl{/augeas/save} node in the Augeas tree\footnoteref{sec:save_node}.

\section{Locating nodes in files}

\label{sec:locating_nodes} \index{augtool!options!--span} \index{Flags!\textsc{aug\_enable\_span}}

The span metadata were added in Augeas 0.8.0. For performance reasons, they are not activated by default. This functionality can be activated by the \verb!AUG_ENABLE_SPAN! flag or the \verb!--span! flag in \verb!augtool!.

You can see if the \verb!span! functionality is activated in the current session by looking at the \nolinkurl{/augeas/span} node\footnoteref{sec:span_node}:

\index{Metadata!\slash{}augeas\slash{}span}

\begin{minted}[fontsize=\scriptsize]{augtool-shell}
augtool> get /augeas/span
/augeas/span = enable
\end{minted}

The data are then available via the \verb!span! command in \verb!augtool!:

\index{Commands!span}

\begin{listing}
  \begin{minted}[fontsize=\scriptsize]{console}
$ augtool --span
  \end{minted}
  \begin{minted}[fontsize=\scriptsize]{augtool-shell}
augtool> get /files/etc/ntp.conf/driftfile
/files/etc/ntp.conf/driftfile = /var/lib/ntp/ntp.drift
augtool> span /files/etc/ntp.conf/driftfile
/etc/ntp.conf label=(67:76) value=(77:99) span=(67,100)
augtool> quit
  \end{minted}
  \begin{minted}[fontsize=\scriptsize]{console}
$ head -c100 /etc/ntp.conf  | tail -c+67

driftfile /var/lib/ntp/ntp.drift
  \end{minted}
  \caption{Getting the position of a node with span}
  \label{lst:span_ntp}
\end{listing}

Listing \ref{lst:span_ntp} indicates that:

\begin{itemize}
\item
  The \verb!driftfile! label was found in the file between positions 67 and 76. This also means that \verb!driftfile! is a dynamic key, not a static label\footnoteref{chap:writing_lenses};
\item
  The value of the \verb!driftfile! node was found between positions 77 and 99 in the file;
\item
  The whole span of the node is between positions 67 and 100 in the file. The span is one character further than the value, since the \verb!\n! character is considered part of the lens matching the node, but is excluded from the value.
\end{itemize}
\section{Scripting with augtool}

\index{augtool!scripting}

In addition to running as an interactive shell, \verb!augtool! can take commands from the command line or STDIN:

\index{Commands!ls} \index{augtool!piping}

\begin{minted}[fontsize=\scriptsize]{console}
$ augtool ls /files
etc/ = (none)
$ echo "ls /files/" | augtool
etc/ = (none)
\end{minted}

This allows to write shell scripts that send commands to \verb!augtool!.

\index{Commands!set} \index{Commands!save} \index{augtool!piping}

\begin{listing}[H]
  \inputminted[linenos,frame=leftline,fontsize=\scriptsize]{bash}{../listings/augtool_wrap.sh}
  \caption{Piping commands to augtool in a bash script}
  \label{lst:augtool_wrap}
\end{listing}

\index{augtool!options!--autosave}

\begin{quote}
\info{The \texttt{--autosave} option in \texttt{augtool} allows you to ommit the \texttt{save} command.}
\end{quote}

Listing \ref{lst:augtool_wrap} shows an example of a bash script wrapping around \verb!augtool!. Lines 2 through 5 define the wrapping function \verb!do_augtool! which is then called on line 7. Commands are separated with \verb!\n! so they get passed line by line to \verb!augtool! through \verb!echo -e!.


\subsection{Using augtool as an interpreter}

\verb!augtool! can also take commands from a file:

\index{augtool!options!--file} \index{Commands!ls}

\begin{listing}
  \begin{minted}[fontsize=\scriptsize]{console}
$ cat commands.augtool
  \end{minted}
  \begin{minted}[fontsize=\scriptsize]{augtool}
ls "/files"
  \end{minted}
  \begin{minted}[fontsize=\scriptsize]{console}
$ augtool --file commands.augtool
  \end{minted}
  \begin{minted}[fontsize=\scriptsize]{augtool-shell}
etc/ = (none)
  \end{minted}
  \caption{\texttt{augtool} takes a command file as argument}
  \label{lst:augtool_file_arg}
\end{listing}

This allows to use \verb!augtool! as a script interpreter in a shebang and write self-executable \verb!augtool! scripts:

\begin{listing}
  \begin{minted}[fontsize=\scriptsize]{console}
$ cat commands.augtool
  \end{minted}
  \begin{minted}[fontsize=\scriptsize]{augtool}
#!/usr/bin/augtool -f
ls "/files"
  \end{minted}
  \begin{minted}[fontsize=\scriptsize]{console}
$ chmod +x commands.augtool
$ ./commands.augtool
  \end{minted}
  \begin{minted}[fontsize=\scriptsize]{augtool-shell}
etc/ = (none)
  \end{minted}
  \caption{Using \texttt{augtool} as an interpreter}
  \label{lst:augtool_interpreter}
\end{listing}

\subsection{Dropping into an interactive session}

When \verb!augtool! takes commands from the command line, STDIN or a file, it doesn't start an interactive session. If you wish to pass commands to \verb!augtool! for preprocessing and run an interactive command afterwards, you can use the \verb!--interactive! flag:

\index{augtool!piping} \index{Commands!get} \index{augtool!options!--interactive}

\begin{listing}
  \begin{minted}[fontsize=\scriptsize]{console}
$ echo "set /files/etc/hosts/1/canonical alice" | augtool --interactive
  \end{minted}
  \begin{minted}[fontsize=\scriptsize]{augtool-shell}
augtool> get /files/etc/hosts/1/canonical
/files/etc/hosts/1/canonical = alice
  \end{minted}
  \caption{Setting a single value in \texttt{augtool}}
  \label{lst:augtool_set_single}
\end{listing}

\begin{quote}
\info{The \texttt{--interactive} option only works for STDIN and file input.}

\end{quote}
This option also allows you to make scripts that set up an environment and drop you in an interactive shell:

\index{augtool!options!--file} \index{Commands!get} \index{Commands!set} \index{Commands!quit}

\begin{minted}[fontsize=\scriptsize]{console}
$ cat shell.augtool
#!/usr/bin/augtool -if
\end{minted}
\begin{minted}[fontsize=\scriptsize]{augtool}
set /files/etc/hosts/1/canonical alice
\end{minted}
\begin{minted}[fontsize=\scriptsize]{console}
$ chmod +x shell.augtool
$ ./shell.augtool
\end{minted}
\begin{minted}[fontsize=\scriptsize]{augtool-shell}
augtool> get /files/etc/hosts/1/canonical
/files/etc/hosts/1/canonical = alice
augtool> quit
\end{minted}

\begin{quote}
\info{Only concatenated short options can be used in shebangs, hence the use of \texttt{-if}.}
\end{quote}

\chapter{Bidirectional transformations}

\label{chap:bx}

Augeas uses files called lenses, written in a specific language for Augeas, which is similar to OCaml. Lenses are programs that are said to be bidirectional.

\section{The Need for Bidirectional Transformations}

Traditional programs take data as input and produce data as output, but cannot use the same code to go from the output data back to the input data. In other words, traditional programs are not bidirectional, they work only in one direction.

For example, the following transformation:

\consolecode@$ echo "I have food in my fridge" | sed -e "s/foo/bar/g"@
\consolecode@I have bard in my fridge@

transforms all \verb!foo! occurrence into \verb!bar! in the string, but it cannot go back from a \verb!bar! string to a \verb!foo! string.

The need to transform from one format and back is quite common, and has traditionally been addressed by writing two programs, one for each direction of the transformation.

In recent years, the Harmony Project\footnote{\url{http://www.seas.upenn.edu/~harmony/}} has been working on the mathematical conditions for programs to be bidirectional, or even bijective. They came up with a language called Boomerang, which implements their theory.

Other projects such as biXid\footnote{\url{http://arbre.is.s.u-tokyo.ac.jp/~hahosoya/papers/bixid.pdf}} or XSugar\footnote{\url{http://www.brics.dk/xsugar/}} have also been working on this same goal concurrently.

XSugar provides a way to transform between XML and non-XML data models, while biXid allows transformations between two XML data models.

\section{A Bit of Theory}

\todo{This might be a part for Francis}
\todo{Also see with Nate if you can use/adapt his schemas}



\subsection{What is a bidirectional transormation}


\subsection{Lenses}

\missingfigure{Lenses}

\todo{Explain about lenses, bidirectional transformations (see Nate's schemas).}

\subsection{Identity and conditions of bidirectionality}

\todo{Give simple identify rules for bidirectional transformations}



\section{Bidirectional transformations in Augeas}


\subsection{Augeas lenses}


\subsection{Lenses typechecking}


\subsection{The case of recursive lenses}



\chapter{Path Expressions}

\index{Path expressions} \index{Tree!\slash{}files|see{Path expressions}}

Augeas maps configuration files into a tree, and lets you access this tree using XPath expressions. In this chapter, we will inspect the various XPath expressions offered by Augeas, and give examples of what you can achieve with them.

\section{Generalities on XPath expressions}

XPath expressions are an XML parsing and modifying facility.

\section{Using globs}

When you write XPath expressions, you might want to match generic nodes or nodes at any level of the tree. There are two operators for that:

\begin{itemize}
\item
  \verb!*! as a node name matches any node;
\item
  \nolinkurl{//} matches on any sublevel of the tree.
\end{itemize}
Examples:

\begin{verbatim}
/files/etc/hosts/*
\end{verbatim}
will match all children nodes of the \nolinkurl{/files/etc/hosts} node.

\begin{verbatim}
/files/etc/hosts//canonical
\end{verbatim}
will match all \verb!canonical! nodes under the \nolinkurl{/files/etc/hosts} node, at any sublevel.

\index{Metadata!\slash{}augeas\slash{}error}

\begin{verbatim}
/augeas//error
\end{verbatim}
will match all \verb!error! nodes at any sublevel under the \nolinkurl{/augeas} node.

\section{Conditionals}

Filtering on node names is often not enough to find what you want. You will often wish to find nodes defined by their value or subnodes. XPath offers a syntax of conditionals using square brackets.

Examples:

\begin{verbatim}
/files/etc/hosts/*[canonical = "alice"]
\end{verbatim}
will match the children nodes of \nolinkurl{/files/etc/hosts} that have a \verb!canonical! subnode with value \verb!alice!.

\begin{verbatim}
/files/etc/hosts/*/canonical[. = "alice"]
\end{verbatim}
will match \verb!canonical! nodes two levels under the \nolinkurl{/files/etc/hosts} node that have value \verb!alice!.

\begin{quote}
\includegraphics{../images/info.png} \emph{In contrast to most XML trees, the Augeas tree contains no attributes, but only nodes with values and children. For this reason, it doesn't use conditional syntaxes featuring the \texttt{@} prefix, which is common to many standard XPath queries.}

\end{quote}
Conditionals can be combined. See these examples:

\begin{verbatim}
/files/etc/hosts/*[ipaddr = "127.0.0.1"][canonical = "alice"]
\end{verbatim}
will match the children nodes of \nolinkurl{/files/etc/hosts} that have both a \verb!ipaddr! sudnode with value \verb!127.0.0.1! and a \verb!canonical! subnode with value \verb!alice!.

\section{Union of paths}

You can use \verb!|! to achieve the union of two paths:

\begin{verbatim}
augtool> match '/files/etc/fstab | /files/etc/hosts'
\end{verbatim}
will return the nodes matching \nolinkurl{/files/etc/fstab} as well as the ones matching \nolinkurl{/files/etc/hosts}.

\section{Functions}

\index{Path expressions!functions|(}

To enrich the filtering you can achieve with conditionals, Augeas provides a set of functions which can be used in conditional context.

\subsection{The last() function}

\index{Path expressions!functions!last()}

\subsection{The position() function}

\index{Path expressions!functions!position()}

\subsection{The label() function}

\index{Path expressions!functions!label()}

\subsection{The count() function}

\index{Path expressions!functions!count()}

\subsection{The regexp() function}

\index{Path expressions!functions!regexp()}

\index{Path expressions!functions|)}

\section{Node references}

In addition to functions, it is often necessary to refer to nodes relatively as you build complex XPath expressions. Augeas provides special node references for that.

\section{Using variables in paths}

\label{sec:variables} \index{Path expressions!variables|(}

Augeas provides two ways to declare variables.

\subsection{defvar}

\index{Path expressions!variables!defvar} \index{Commands!defvar}

\subsection{defnode}

\index{Path expressions!variables!defnode} \index{Commands!defnode}

\index{Path expressions!variables|)}

\subsection{Using variables to express conditionals}

\section{Ensuring idempotence}

\label{sec:ensuring_idempotence} \index{Path expressions!idempotence}

\begin{verbatim}
augtool> set '/files/etc/php.ini/PHP/extension[. = "foo.so"]' foo.so
\end{verbatim}


\chapter{Using the C API and Bindings}

\label{chap:api}
\index{API|(} \index{API!calls|see{Calls}}

\chquote{He cleaned in one day the ox dung of King Augeas,\\
Jove helping him for the most part.\\
By letting in a river\\
he washed away all the dung.}{Fabula 30}{Hyginus}

So far, our examples have been done using \verb!augtool!, the CLI interface to Augeas. However, Augeas is first and foremost a C library.

\section{Using the C API}

\index{API!C API}

\section{API Flags}

\index{Flags} \index{API!flags|see{Flags}}

\section{Using Bindings}

\index{API!bindings|(}

\subsection{Haskell bindings}

\index{API!bindings!Haskell}

\subsection{Java bindings}

\index{API!bindings!Java}

\subsection{Perl Bindings}

\index{API!bindings!Perl}

\subsection{PHP bindings}

\index{API!bindings!PHP}

\subsection{Python Bindings}

\index{API!bindings!Python}

\subsubsection{Installation}

\subsubsection{Initialization}

Synopsis:

\begin{python}[]
    def __init__(self, root=None, loadpath=None, flags=NONE)
\end{python}

Initialize the library.

Use \verb!root! as the filesystem root. If \verb!root! is None, use the value of the environment variable AUGEAS\_ROOT. If that doesn't exist either, use \nolinkurl{/}.

\verb!loadpath! is a colon-spearated list of directories that modules should be searched in. This is in addition to the standard load path and the directories in \verb!AUGEAS_LENS_LIB!.

\verb!flags! is a bitmask made up of values from \verb!AUG_FLAGS!.

Example:

\begin{python}[]
    import augeas
    a = augeas.Augeas(root="fakeroot")
\end{python}

\subsubsection{The get method}

Synopsis:

\begin{python}[]
    def get(self, path)
\end{python}

Lookup the value associated with \verb!path!. Returns the value at the path specified. It is an error if more than one node matches \verb!path!.

Example:

\begin{python}[]
    val = a.get("/files/etc/ftab/1/canonical")
\end{python}

\subsection{Ruby Bindings}

\index{API!bindings!Ruby}

\index{API!bindings|)}

\index{API|)}



\chapter{Augeas metadata}

\index{Metadata|(}
\index{Tree!\slash{}augeas|see{Metadata}}

We have seen earlier that the \verb!/augeas! top node exposes Augeas metadata which can be parsed and modified in the same fashion as the \verb!/files! data. This chapter will focus on documenting the various parts of the \verb!/augeas! tree and their functions.

\section{The root node}

\index{Metadata!\slash{}augeas\slash{}root} \index{augtool!options!--root} \index{Environment variables!\textsc{augeas\_root}}

The \verb!/augeas/root! node contains the root of the Augeas tree. This is the variable which can be set via either the \verb!AUGEAS_ROOT! environment variable or the \verb!--root! option to \verb!augtool!.

\index{Commands!print}

Example:

\begin{listing}
  \begin{minted}{console}
$ augtool --root fakeroot
  \end{minted}
  \begin{minted}{augtool-shell}
augtool> print /augeas/root
/augeas/root = "fakeroot/"
  \end{minted}
  \label{lst:metadata_root}
  \caption{Inspecting /augeas/root}
\end{listing}

\begin{quote}
\includegraphics{../images/info.png} \emph{As of Augeas 0.8.0, this node is purely informative: changing its value has no effect on the way Augeas works.}

\end{quote}
\section{The version tree}

\index{Metadata!\slash{}augeas\slash{}version}

\verb!/augeas/version! is a tree which contains several informations:

\begin{itemize}
\item
  The top node has the version of Augeas as its value ;
\item
  The \verb!save! node contains \verb!mode! nodes which list the known saving modes for this version of Augeas ;
\item
  The \verb!defvar! node contains \textbf{what exactly??}.
\end{itemize}
\index{Commands!print}


\begin{listing}
  \begin{minted}{augtool-shell}
augtool> print /augeas/version/
/augeas/version = "0.8.0"
/augeas/version/save
/augeas/version/save/mode[1] = "backup"
/augeas/version/save/mode[2] = "newfile"
/augeas/version/save/mode[3] = "noop"
/augeas/version/save/mode[4] = "overwrite"
/augeas/version/defvar
/augeas/version/defvar/expr
  \end{minted}
  \label{lst:metadata_version}
  \caption{Inspecting /augeas/version}
\end{listing}


\section{The save node}

\label{sec:save_node} \index{Metadata!\slash{}augeas\slash{}save}

The \verb!/augeas/save! node contains the saving mode used by Augeas for the session. The value of this node must be one of the values listed in the \verb!/augeas/version/save/mode! nodes.

If this node is modified during the session, it will affect the behaviour of the \verb!save! call whenever it is executed.

\section{The load tree}

\index{Metadata!\slash{}augeas\slash{}load}

The \verb!/augeas/load! tree contains the lenses metadata. For each lens loaded in the Augeas session, it lists 3 types of nodes:

\begin{itemize}
\item
  a \verb!lens! node, which specifies the name of the module used by this lens ;
\item
  \verb!incl! nodes for each inclusion path to files recognized by this lens ;
\item
  \verb!excl! nodes for each path to be excluded from this lens.
\end{itemize}
\index{Commands!print}

\begin{listing}
  \begin{minted}{augtool-shell}
augtool> print /augeas/load/Pam/
/augeas/load/Pam
/augeas/load/Pam/lens = "@Pam"
/augeas/load/Pam/incl = "/etc/pam.d/*"
/augeas/load/Pam/excl[1] = "*.augnew"
/augeas/load/Pam/excl[2] = "*.augsave"
/augeas/load/Pam/excl[3] = "*.dpkg-dist"
/augeas/load/Pam/excl[4] = "*.dpkg-bak"
/augeas/load/Pam/excl[5] = "*.dpkg-new"
/augeas/load/Pam/excl[6] = "*.dpkg-old"
/augeas/load/Pam/excl[7] = "*.rpmsave"
/augeas/load/Pam/excl[8] = "*.rpmnew"
/augeas/load/Pam/excl[9] = "*~"
  \end{minted}
  \label{lst:metadata_load_pam}
  \caption{Listing metadata for the Pam module}
\end{listing}


This tree can be manipulated to fine tune the lenses known by Augeas for a session, as well as the files parsed in the session. When the \verb!/augeas/load! tree is modified, you have to call \verb!load! again for the changes to take effect.

Let us look at some use cases.

\subsection{Using only one lens}

It is common to use Augeas to modify only one file. In that case you know exactly which lens you want to use and on which file. For performance reasons, you might want to narrow the lenses and files Augeas knows about. For example, if you want to only modify \verb!/etc/fstab!, using the \verb!Fstab! lens. In order to do that, we can start \verb!augtool! without loading any lenses:

\index{augtool!options!--noautoload}

\begin{listing}
  \begin{minted}{console}
$ augtool --noautoload
  \end{minted}
  \begin{minted}{augtool-shell}
augtool> print /augeas/load
/augeas/load
  \end{minted}
  \label{lst:metadata_noautoload}
  \caption{The effect of --noautoload on /augeas/load}
\end{listing}


\index{Flags!\textsc{aug\_no\_modl\_autoload}}

\begin{quote}
\includegraphics{../images/info.png} \emph{This can also be achieved using the \texttt{AUG\_NO\_MODL\_AUTOLOAD} flag with the API}

\end{quote}
The \verb!print! command shows us that no lenses are known in the session. We can now tell Augeas to load the \verb!Fstab! lens and to include \verb!/etc/fstab! for it:

\index{Commands!set} \index{Commands!print}

\begin{listing}
  \begin{minted}{augtool-shell}
augtool> set /augeas/load/Fstab/lens "Fstab.lns"
augtool> set /augeas/load/Fstab/incl "/etc/fstab"
augtool> print /augeas/load
/augeas/load
/augeas/load/Fstab
/augeas/load/Fstab/lens = "Fstab.lns"
/augeas/load/Fstab/incl = "/etc/fstab"
  \end{minted}
  \label{lst:metadata_setting_lens_manually}
  \caption{Setting the Fstab lens manually in /augeas/load}
\end{listing}

We can now call \verb!load! and list the files in \verb!/files/etc!:

\index{Commands!load} \index{Commands!ls}

\begin{listing}
  \begin{minted}{augtool-shell}
augtool> load
augtool> ls /files/etc
fstab/ = (none)
  \end{minted}
  \label{lst:metadata_call_load}
  \caption{Loading files manually}
\end{listing}

\begin{quote}
\includegraphics{../images/info.png} \emph{Lenses loaded automatically have a \texttt{lens} statement which begins with a \texttt{@}, such a \texttt{@Fstab}. When you set the lens manually however, you have to specify the lens to use, for example \texttt{Fstab.lns}. See chapter \ref{chap:writing_lenses} on page \pageref{chap:writing_lenses} for more information on writing lenses.}

\end{quote}
\subsection{Parsing a specific file}

Augeas lenses have hardcoded lists of files they know about. For example the \verb!Fstab! lens has an include statement for \verb!/etc/fstab! hardcoded in \verb!fstab.aug!. While Augeas attempts to cover the most common needs for inclusions, it cannot know about all files you are using. Some lenses don't even have default include statements because no common files are known to use them. This is the case of the \verb!Json! lens, which is useful but applies to no common configuration file.

So how do you go about using the \verb!Json! lens on a JSON file? You can modify the \verb!/augeas/load! tree for that. For example if you have a \verb!foo.json! file in your current directory, you could do the following:

\index{augtool!options!--root} \index{Commands!set} \index{Commands!load} \index{Commands!ls}

\begin{listing}
  \begin{minted}{console}
$ augtool --root .
  \end{minted}
  \begin{minted}{augtool-shell}
augtool> set /augeas/load/Json/incl "/foo.json"
augtool> load
augtool> ls /files
foo.json/ = (none)
  \end{minted}
  \label{lst:metadata_json_lens}
  \caption{Using the Json lens with /augeas/load}
\end{listing}

\begin{quote}
\includegraphics{../images/info.png} \emph{This technique can be combined with the above to load only the \texttt{Json} module}

\end{quote}
\section{The files tree}

\index{Metadata!\slash{}augeas\slash{}files|(}

The \verb!/augeas/files! provides metadata about the files parsed by Augeas. The paths in this tree mirror thoses of the \verb!/files! tree.

For each file, the following nodes may be present.

\subsection{The path node}

\verb!path! is the path to the file data in the \verb!/files! tree ;

\subsection{The mtime node}

\verb!mtime! is the last modification time of the file ;

\subsection{The lens tree}

The \verb!lens! tree indicates the lens used to parse this file, as specified in the \verb!/augeas/load! tree (see above). The \verb!lens/info! node gives the path to the lens module (physically), as well as the position of the lens declaration in the file.

\subsection{The error tree}

When Augeas fails to parse a file, the parsing error is listed here.

This tree contains several nodes:

\begin{itemize}
\item
  \verb!pos! is the position in the file, relative to the beginning, where Augeas failed to parse ;
\item
  \verb!line! is the line in the file where Augeas failed to parse ;
\item
  \verb!char! is the character of the line where Augeas failed to parse ;
\item
  \verb!lens! is the lens that failed to parse. It is usually the same as as \verb!lens/info! node listed above ;
\item
  \verb!message! is the error message yielded by Augeas.
\end{itemize}
For more information on interpreting the error messages, see chapter \ref{chap:troubleshooting} on page \pageref{chap:troubleshooting}.

\subsection{Example}

\begin{listing}
  \begin{minted}{console}
$ augtool 
  \end{minted}
  \begin{minted}{augtool-shell}
augtool> print /augeas/files/etc/ldap.conf/
/augeas/files/etc/ldap.conf
/augeas/files/etc/ldap.conf/path = "/files/etc/ldap.conf"
/augeas/files/etc/ldap.conf/mtime = "1298365882"
/augeas/files/etc/ldap.conf/lens = "@Spacevars"
/augeas/files/etc/ldap.conf/lens/info = "/usr/share/augeas/lenses/dist/spacevars.aug:37.23-.46:"
/augeas/files/etc/ldap.conf/error = "parse_failed"
/augeas/files/etc/ldap.conf/error/pos = "9510"
/augeas/files/etc/ldap.conf/error/line = "310"
/augeas/files/etc/ldap.conf/error/char = "0"
/augeas/files/etc/ldap.conf/error/lens = "/usr/share/augeas/lenses/dist/spacevars.aug:37.23-.46:"
/augeas/files/etc/ldap.conf/error/message = "Iterated lens matched less than it should"
  \end{minted}
  \label{lst:metadata_ldap_conf}
  \caption{Inspecting ldap.conf metadata}
\end{listing}

In the example above, we see the that \verb!/etc/ldap.conf! uses the \verb!@Spacevars! lens, located in \verb!spacevars.aug! on line 37, between characters 23 et 46.

The parsing of \verb!/etc/ldap.conf! failed on position 9510, which located in beginning of line 310. The error message indicates that the file could not be fully parsed.

\index{Metadata!\slash{}augeas\slash{}files|)}

\section{The variables tree}

\index{Metadata!\slash{}augeas\slash{}variables} \index{Path expressions!variables!defvar}

When you set variables in Augeas\footnote{See \emph{using variables in paths} on page \pageref{sec:variables}} the paths of the variables are recorded here.

\index{Commands!print} \index{Commands!defvar}

Example:

\begin{listing}
  \begin{minted}{augtool-shell}
augtool> defvar l /augeas/files/etc/ldap.conf/
augtool> print /augeas/variables/
/augeas/variables
/augeas/variables/l = "/augeas/files/etc/ldap.conf"
  \end{minted}
  \label{lst:metadata_defvar}
  \caption{Defined variables are listed in /augeas/variables}
\end{listing}

\begin{quote}
\includegraphics{../images/info.png} \emph{As of Augeas 0.8.0, this node is purely informative: changing its value has no effect on the way Augeas works.}

\end{quote}
\section{The span node}

\label{sec:span_node} \index{Metadata!\slash{}augeas\slash{}span} \index{augtool!options!--span} \index{Flags!\textsc{aug\_enable\_span}}

The \verb!/augeas/span! node indicates whether the \verb!span! functionality\footnote{See \emph{locating nodes in files} on page \pageref{sec:locating_nodes}} is activated in the session.

\index{Metadata|)}

\chapter{Using Augeas in Puppet}

\index{Puppet|(}
Because Augeas is a configuration API, it fits right into tools that are made for configuration management. One of the most widely used of these tools in the open-source world is Puppet, and Augeas has been available as a native type in Puppet since version 0.24.7.

\index{API!bindings!Ruby}
Since Puppet is written in Ruby, the Augeas Puppet type makes use of the Ruby bindings for Augeas.

\section{The Augeas type}

Puppet provides a native Augeas type since version 0.24.7.

\index{Commands!set} The Augeas type in Puppet takes a list of commands labeled ``changes''. A simple example is the following:

\begin{verbatim}
augeas { "hosts_alice":
   changes => [
      "set /files/etc/hosts/1/canonical alice",
   ],
}
\end{verbatim}
The \verb!changes! attribute is an array of Augeas commands, similar to what you would pass to \verb!augtool!.

\begin{quote}
\includegraphics{../images/info.png} \emph{It is recommended to use \texttt{augtool} to prepare and test the commands before you use them in Puppet.}

\end{quote}
\index{Commands!save} Each call to the Augeas type starts a new Augeas session. The \verb!save! call is ran automatically at the end of each session.

\section{Setting a context}

\section{Proper quoting}

While quoting in \verb!augtool! is strict, quoting in Puppet can be tricky.

\textbf{TODO}: explain this

\section{Puppet and idempotence}

Idempotence is very important in configuration management tools such as Puppet. The Augeas type provides a \verb!onlyif! statement to make it easy to ensure that Augeas is only called when necessary.

\begin{verbatim}
augeas { "hosts_alice":
   context => "/files/etc/hosts/1",
   changes => [
      "set canonical alice",
   ],
   onlyif => "match canonical[. = 'alice']" size == 0",
}
\end{verbatim}

\begin{quote}
\includegraphics{../images/info.png} \emph{For proper idempotence, this statement has to be coupled with the methods described earlier\footnoteref{sec:ensuring_idempotence}.}
\end{quote}

\index{Puppet|)}

\chapter{Stock Modules Reference}

Augeas comes with an extensive collection of lenses to parse common configuration files. This chapter will cover the most common of these lenses and provide a reference and examples for them.

This list is not complete. The complete documentation of the modules can be obtained by building it from the Augeas source\footnoteref{sec:installing_augeas}.

\section{The Access Module}

Description of the module

\section{The Aliases Module}

Description of the module

\section{The Approx Module}

Description of the module

\section{The AptPreferences Module}

Description of the module

\section{The Aptsources Module}

Description of the module

\section{The Cron Module}

Description of the module

\section{The Crypttab Module}

Description of the module

\section{The Dhclient Module}

Description of the module

\section{The Dhcpd Module}

Description of the module

\section{The Dnsmasq Module}

Description of the module

\section{The Ethers Module}

Description of the module

\section{The Exports Module}

Description of the module

\section{The Fstab Module}

The \verb!Fstab! modules parses \nolinkurl{/etc/fstab} and \nolinkurl{/etc/mtab}.

\subsection{Anatomy}

The root nodes are either \verb!#comment! nodes or \verb!seq! nodes.

Each of the \verb!seq! nodes correspond to a line in the file, and has the following nodes:

\begin{itemize}
\item
  A \verb!spec! node, the block special device or remote filesystem to be mounted ;
\item
  A \verb!file! node, the mount point for the file system ;
\item
  A \verb!vfstype! node, the type of filesystem ;
\item
  One or more \verb!opt! nodes, the mount options associated with the filesystem ;
\item
  An optional \verb!dump! node, the parameter for the \verb!dump(8)! command ;
\item
  An optional \verb!passno! node, the parameter for the \verb!fsck(8)! command.
\end{itemize}
See \verb!man fstab! for more information on these fields.

\subsection{Sample tree}

\begin{minted}[fontsize=\scriptsize]{augtool-shell}
/files/etc/fstab
/files/etc/fstab/#comment[1] = "/etc/fstab: static file system information."
/files/etc/fstab/#comment[2] = "Use 'blkid -o value -s UUID' to print the universally unique identifier"
/files/etc/fstab/#comment[3] = "for a device; this may be used with UUID= as a more robust way to name"
/files/etc/fstab/#comment[4] = "devices that works even if disks are added and removed. See fstab(5)."
/files/etc/fstab/#comment[5] = "<file system> <mount point>   <type>  <options>       <dump>  <pass>"
/files/etc/fstab/1
/files/etc/fstab/1/spec = "proc"
/files/etc/fstab/1/file = "/proc"
/files/etc/fstab/1/vfstype = "proc"
/files/etc/fstab/1/opt[1] = "nodev"
/files/etc/fstab/1/opt[2] = "noexec"
/files/etc/fstab/1/opt[3] = "nosuid"
/files/etc/fstab/1/dump = "0"
/files/etc/fstab/1/passno = "0"
/files/etc/fstab/#comment[6] = "/ was on /dev/sda1 during installation"
/files/etc/fstab/2
/files/etc/fstab/2/spec = "UUID=4cbb4f80-45f9-4e46-a076-8ec1124f4835"
/files/etc/fstab/2/file = "/"
/files/etc/fstab/2/vfstype = "ext3"
/files/etc/fstab/2/opt = "errors"
/files/etc/fstab/2/opt/value = "remount-ro"
/files/etc/fstab/2/dump = "0"
/files/etc/fstab/2/passno = "1"
/files/etc/fstab/3
/files/etc/fstab/3/spec = "/dev/sda2"
/files/etc/fstab/3/file = "none"
/files/etc/fstab/3/vfstype = "swap"
/files/etc/fstab/3/opt = "rw"
/files/etc/fstab/3/dump = "0"
/files/etc/fstab/3/passno = "0"
\end{minted}

\subsection{Examples of path expressions}

\begin{minted}[fontsize=\scriptsize]{augtool}
match '/files/etc/fstab/*[label() != "#comment"]'
\end{minted}

matches all lines except comment lines.

\begin{minted}[fontsize=\scriptsize]{augtool}
match '/files/etc/fstab/*[file = "/"]/spec'
\end{minted}

matches the \verb!spec! node for the filesystem mounted on ``/''.

\begin{minted}[fontsize=\scriptsize]{augtool}
rm '/files/etc/fstab/*/opt[. = "nosuid"]'
\end{minted}

removes all \verb!opt! nodes whose value is \verb!nosuid!.

\section{The Group Module}

Description of the module

\section{The Grub Module}

Description of the module

\section{The Hosts Module}

Description of the module

\section{The Httpd Module}

Description of the module

\section{The Inetd Module}

Description of the module

\section{The Inittab Module}

Description of the module

\section{The Interfaces Module}

Description of the module

\section{The Iptables Module}

Description of the module

\section{The Json Module}

Description of the module

\section{The Limits Module}

Description of the module

\section{The Login\_defs Module}

Description of the module

\section{The Logrotate Module}

Description of the module

\section{The Modprobe Module}

Description of the module

\section{The Modules\_conf Module}

Description of the module

\section{The MySQL Module}

Description of the module

\section{The Nsswitch Module}

Description of the module

\section{The Ntp Module}

Description of the module

\section{The Pam Module}

Description of the module

\section{The Passwd Module}

Description of the module

\section{The Pg\_Hba Module}

Description of the module

\section{The PHP Module}

Description of the module

\section{The Postfix\_Master Module}

Description of the module

\section{The Puppet Module}

Description of the module

\section{The Resolv Module}

Description of the module

\section{The Services Module}

Description of the module

\section{The Shells Module}

Description of the module

\section{The Sshd Module}

Description of the module

\section{The Sudoers Module}

Description of the module

\section{The Sysconfig Module}

Description of the module

\section{The Sysctl Module}

Description of the module

\section{The Syslog Module}

Description of the module

\section{The Xinetd Module}

Description of the module

\section{The Xml Module}

Description of the module

\section{The Yum Module}

Description of the module



\chapter{Writing Your Own Lenses}

\label{chap:writing_lenses} \index{Lenses!writing}

Augeas comes with a set of various lenses which cover most of the basic configuration files on a Unix machine. However, there are so many configuration file formats on Unix systems, that you are very likely to miss one at some point.

Augeas lenses are written in a ML language that is similar to OCaml. The language consists mostly of regexps and operators to combine them.

\section{A simple example}

Since Augeas lenses are mostly a combination of regular expressions that are often complex and fragile, it is safer to consider writing unit tests for each lens to ensure non-regression and confirm that all known cases are met by the lens. Our work will thus begin with the writing of a unit test, which will specify the way we will map the configuration entries to the Augeas tree. For extended information on unit tests, see the end of this chapter.

\subsection{Unit Test}

\textbf{Example of a simple key/value conffilem, step by step}

\subsection{Module}

\textbf{Example of a simple key/value conffile, step by step}

\section{Regular expressions}

The bidirectional nature of the Augeas language imposes strict conditions on the language\footnote{See chapter \ref{chap:bx} on page \pageref{bx}}. This makes complex regular expressions languages such as PCRE hard to implement. For this reason, Augeas only supports POSIX simple regular expressions.

\textbf{Give Examples}

\section{Special keywords}

The Augeas language provides a set of keywords to build lenses.

\subsection{key}

\subsection{label}

\subsection{store}

\subsection{value}

\subsection{seq}

\subsection{rec}

\subsection{square}

\section{Combination Operators}

Augeas lenses are put together by assembling regular expressions with combination operators.

\subsection{Concatenation Operator}

\subsection{Union Operator}

\section{Filters and Autoload}

Augeas lenses need to specify which files they apply to. If they didn't, Augeas would have no way to know which lens to apply to which files. Trying to guess would be a really bad idea. For example, consider a file whose only content is the following:

\begin{minted}{bash}
# this is a comment
\end{minted}

Many lenses are able to parse this line, and will mostly likely map it the same way. However, once a lens has been chosen for the file, the rest of the configuration statements are likely to be very different from one lens to another, so you are almost sure that the lens you chose will be wrong.

Each lenses may have one and only one autoload statement, involving a lens and a filter, such as the following:

\begin{minted}{augeas}
autoload xfm
let lns = ...
let filter = incl "/etc/foo.conf"
let xfm = transform lns filter
\end{minted}


\section{Typechecking lenses}

Augeas comes with a command line tool called \verb!augparse! which can be used to typecheck lenses, checking that they meet the conditions to be used as bidirectional transforms.

\subsection{Typechecking recursive lenses}

\section{Unit tests}

We have mentionned the importance of unit tests in the beginning of this chapter. It is worth repeting it: unit tests are essential to the stability of an Augeas lens. Unit tests need to be well written and kept up-to-date with new features and bug fixes to ensure that the lens continues to work with the files it was written for.

Augeas provides keywords to achieve unit tests in both the get and put directions.

\section{Using Generic Modules}

Augeas provides special modules to ease the writing of lenses.

\subsection{The Util module}

The Util (\verb!util.aug!) module provides definitions of comments, empty lines and other utilities.

\textbf{List functions} and give examples.

\subsection{The Sep module}

The Sep (\verb!sep.aug!) module provides definitions for separators. \textbf{List functions} and give examples.

\begin{quote}
\includegraphics{../images/info.png} \emph{\texttt{Sep.opt\_space} is a synonym for \texttt{Util.indent}. Both are strictly equivalent, but it is clearer to use the former as a separator and the latter as an indentation.}

\end{quote}
\subsection{The Rx module}

The Rx (\verb!rx.aug!) module provides definitions for usual regular expressions. \textbf{List functions} and give examples.

\subsection{The Build module}

The Build (\verb!build.aug!) module provides definitions for usual constructions of regular expression. \textbf{List functions} and give examples.

\subsection{The IniFile module}

INI files are quite standard even on Unix systems. However, there are many different implementations and variations. The Inifile (\verb!inifile.aug!) module provides definitions to ease the writing of lenses for specific INI files. It is used as a basis for lenses such as Php (\verb!php.aug!), MySQL (\verb!mysql.aug!) or Puppet (\verb!puppet.aug!). \textbf{List functions} and give examples.

\section{Using your lens}

Augeas uses a search path to find its lenses. By default, it will search for lenses in \verb!$prefix/share/augeas/lenses! and \verb!/$prefix/share/augeas/lenses/dist!, where \verb!$prefix! is the compilation prefix, usually \verb!/usr!.

The \verb!dist! subdirectory is reserved for stock lenses, while the top directory can be used to store your own lenses.

If you prefer to store your lenses in another place, or just wish to try a new lens without installing it in your system, you can override this search path in several ways.

\subsection{Ignoring the stock modules}

\index{augtool!options!--nostdinc}

In order to ignore the default search path for lenses, you can use the \verb!--nostdinc! flag in \verb!augtool!.

\subsection{Adding your own directory of lenses}

\index{augtool!options!--include} \index{Environment variables!\textsc{augeas\_lens\_lib}}

Directories containing additional lenses can be added to the search path by using the \verb!--include! option in \verb!augtool!, or the \verb!AUGEAS_LENS_LIB! environment variable:

\begin{minted}{console}
$ augtool --include mylenses
\end{minted}



\chapter{Troubleshooting Augeas}

\label{chap:troubleshooting} \index{Lenses!troubleshooting}

The Augeas tree is built using bidirectional grammars called lenses\footnoteref{chap:bx}. The configuration files will not appear in the Augeas tree if the lens responsible for parsing them fails to do so.

In the other direction\footnote{The \verb!put! direction ; \seeref{chap:bx}}, lenses may fail to save a tree back to a configuration file if that tree doesn't fit in the given lens.

Whatever you are trying to troubleshoot, you will most likely benefit from the metadata exposed in the \nolinkurl{/augeas} node at the top of the Augeas tree.

A simple way to list all known errors in an augtool session is to type:

\index{Commands!print} \index{Metadata!\slash{}augeas\slash{}error}

\begin{minted}[fontsize=\scriptsize]{augtool-shell}
augtool> print /augeas//error
\end{minted}

The double slash tells Augeas to search for all subnodes under \nolinkurl{/augeas} whose label matches ``error''. The print command will return all subnodes of the matching nodes, given you the details of the errors.

If you want to see the error on a specific file, you can use the path to that file in the expression. For example, to see the error on \nolinkurl{/etc/fstab}, you can use:

\index{Commands!print} \index{Metadata!\slash{}augeas\slash{}error}

\begin{minted}[fontsize=\scriptsize]{augtool-shell}
augtool> print /augeas/files/etc/fstab/error
\end{minted}

\section{Files don't appear in the tree}

There can be several reasons for a file to not appear in the Augeas tree.

\subsection{No lens for the file}

One possibility is that there is no existing lens for this file, or the lens you expect to parse this file has no filter for this file at this location. See chapter \ref{chap:writing_lenses} on page \pageref{chap:writing_lenses} for more information on writing lenses.

\subsection{UID has no rights to read}

Another possibility is that the Unix UID you are using has no right to see the file. The ``error'' node in the \nolinkurl{/augeas} tree will tell you so, with a message such as:

\index{Metadata!\slash{}augeas\slash{}error}

\begin{minted}[fontsize=\scriptsize]{augtool-shell}
/augeas/files/etc/sudoers/error = "read_failed"
/augeas/files/etc/sudoers/error/message = "Permission denied"
\end{minted}

\subsection{Parsing failed}

The last possibility is that the lens failed to parse part of the file, or the whole file.

Parsing errors are quite common, and there can be several reasons for them:

\begin{itemize}
\item
  The file uses \verb!\r! for newlines. Most lenses, having been made for Unix systems, only recognize \verb!\n! as valid newlines. Getting the file through dos2unix and trying again can confirm this possibility.
\item
  The lens fails to parse a part of the file, for example it doesn't cover a specific case that is valid for this configuration file.
\item
  The lens fails to parse the entire file.
\end{itemize}
In the last two cases, it is important to check that the configuration file is indeed valid. When available, use a command line tool provided with the application owning the configuration file, such as apachectl or visudo:

\begin{minted}[fontsize=\scriptsize]{console}
$ apachectl configtest
$ visudo -c
\end{minted}

Note that when the application owning the configuration file is happy with the file and Augeas is not, it is always safer to consider that Augeas is wrong and that the lens has to be modified, since other users are likely to be in the same situation.

\section{Save failed}

Just as files can fail to be parsed by Augeas, trees can fail to be transformed back into files, too. This prevents Augeas from saving a tree that wouldn't make sense in the configuration file, thus preventing it from breaking configuration files.

\textbf{Explain cases and solutions}

\section{Turning on debug}

\index{Environment variables!\textsc{augeas\_debug}} \index{Environment variables!\textsc{augeas\_debug\_dir}}

Augeas has a debug facility that is turned off by default. Two environment variables control the activate of this functionality: \verb!AUGEAS_DEBUG! and \verb!AUGEAS_DEBUG_DIR!.



\chapter{Contacting the Augeas team}

\chquote{Tell me and I forget.\\
Teach me and I remember.\\
Involve me and I learn.}{}{Benjamin Franklin}

Augeas is an open-source project with an active community of users and developers.

There are several ways to contact the Augeas team:

\begin{itemize}
\item
  The augeas-devel mailing list at https://www.redhat.com/mailman/listinfo/augeas-devel;
\item
  The IRC channel \verb!#augeas! on the Freenode IRC network.
\end{itemize}


\section{Contributing}

You are very welcome to contribute code to the Augeas project.

\subsection{Forking the git repository}

\subsection{Coding style}

\subsection{Sending patches}

\section{Getting support}

\section{Reporting bugs}




%GFDL
\appendix
\chapter*{\rlap{GNU Free Documentation License}}
\phantomsection  % so hyperref creates bookmarks
\addcontentsline{toc}{chapter}{GNU Free Documentation License}
%\label{label_fdl}

 \begin{center}

       Version 1.3, 3 November 2008


 Copyright \copyright{} 2000, 2001, 2002, 2007, 2008  Free Software Foundation, Inc.
 
 \bigskip
 
     \url{http://fsf.org/}
  
 \bigskip
 
 Everyone is permitted to copy and distribute verbatim copies
 of this license document, but changing it is not allowed.
\end{center}


\begin{center}
{\bf\large Preamble}
\end{center}

The purpose of this License is to make a manual, textbook, or other
functional and useful document ``free'' in the sense of freedom: to
assure everyone the effective freedom to copy and redistribute it,
with or without modifying it, either commercially or noncommercially.
Secondarily, this License preserves for the author and publisher a way
to get credit for their work, while not being considered responsible
for modifications made by others.

This License is a kind of ``copyleft'', which means that derivative
works of the document must themselves be free in the same sense.  It
complements the GNU General Public License, which is a copyleft
license designed for free software.

We have designed this License in order to use it for manuals for free
software, because free software needs free documentation: a free
program should come with manuals providing the same freedoms that the
software does.  But this License is not limited to software manuals;
it can be used for any textual work, regardless of subject matter or
whether it is published as a printed book.  We recommend this License
principally for works whose purpose is instruction or reference.


\begin{center}
{\Large\bf 1. APPLICABILITY AND DEFINITIONS\par}
\phantomsection
\addcontentsline{toc}{section}{1. APPLICABILITY AND DEFINITIONS}
\end{center}

This License applies to any manual or other work, in any medium, that
contains a notice placed by the copyright holder saying it can be
distributed under the terms of this License.  Such a notice grants a
world-wide, royalty-free license, unlimited in duration, to use that
work under the conditions stated herein.  The ``\textbf{Document}'', below,
refers to any such manual or work.  Any member of the public is a
licensee, and is addressed as ``\textbf{you}''.  You accept the license if you
copy, modify or distribute the work in a way requiring permission
under copyright law.

A ``\textbf{Modified Version}'' of the Document means any work containing the
Document or a portion of it, either copied verbatim, or with
modifications and/or translated into another language.

A ``\textbf{Secondary Section}'' is a named appendix or a front-matter section of
the Document that deals exclusively with the relationship of the
publishers or authors of the Document to the Document's overall subject
(or to related matters) and contains nothing that could fall directly
within that overall subject.  (Thus, if the Document is in part a
textbook of mathematics, a Secondary Section may not explain any
mathematics.)  The relationship could be a matter of historical
connection with the subject or with related matters, or of legal,
commercial, philosophical, ethical or political position regarding
them.

The ``\textbf{Invariant Sections}'' are certain Secondary Sections whose titles
are designated, as being those of Invariant Sections, in the notice
that says that the Document is released under this License.  If a
section does not fit the above definition of Secondary then it is not
allowed to be designated as Invariant.  The Document may contain zero
Invariant Sections.  If the Document does not identify any Invariant
Sections then there are none.

The ``\textbf{Cover Texts}'' are certain short passages of text that are listed,
as Front-Cover Texts or Back-Cover Texts, in the notice that says that
the Document is released under this License.  A Front-Cover Text may
be at most 5 words, and a Back-Cover Text may be at most 25 words.

A ``\textbf{Transparent}'' copy of the Document means a machine-readable copy,
represented in a format whose specification is available to the
general public, that is suitable for revising the document
straightforwardly with generic text editors or (for images composed of
pixels) generic paint programs or (for drawings) some widely available
drawing editor, and that is suitable for input to text formatters or
for automatic translation to a variety of formats suitable for input
to text formatters.  A copy made in an otherwise Transparent file
format whose markup, or absence of markup, has been arranged to thwart
or discourage subsequent modification by readers is not Transparent.
An image format is not Transparent if used for any substantial amount
of text.  A copy that is not ``Transparent'' is called ``\textbf{Opaque}''.

Examples of suitable formats for Transparent copies include plain
ASCII without markup, Texinfo input format, LaTeX input format, SGML
or XML using a publicly available DTD, and standard-conforming simple
HTML, PostScript or PDF designed for human modification.  Examples of
transparent image formats include PNG, XCF and JPG.  Opaque formats
include proprietary formats that can be read and edited only by
proprietary word processors, SGML or XML for which the DTD and/or
processing tools are not generally available, and the
machine-generated HTML, PostScript or PDF produced by some word
processors for output purposes only.

The ``\textbf{Title Page}'' means, for a printed book, the title page itself,
plus such following pages as are needed to hold, legibly, the material
this License requires to appear in the title page.  For works in
formats which do not have any title page as such, ``Title Page'' means
the text near the most prominent appearance of the work's title,
preceding the beginning of the body of the text.

The ``\textbf{publisher}'' means any person or entity that distributes
copies of the Document to the public.

A section ``\textbf{Entitled XYZ}'' means a named subunit of the Document whose
title either is precisely XYZ or contains XYZ in parentheses following
text that translates XYZ in another language.  (Here XYZ stands for a
specific section name mentioned below, such as ``\textbf{Acknowledgements}'',
``\textbf{Dedications}'', ``\textbf{Endorsements}'', or ``\textbf{History}''.)  
To ``\textbf{Preserve the Title}''
of such a section when you modify the Document means that it remains a
section ``Entitled XYZ'' according to this definition.

The Document may include Warranty Disclaimers next to the notice which
states that this License applies to the Document.  These Warranty
Disclaimers are considered to be included by reference in this
License, but only as regards disclaiming warranties: any other
implication that these Warranty Disclaimers may have is void and has
no effect on the meaning of this License.


\begin{center}
{\Large\bf 2. VERBATIM COPYING\par}
\phantomsection
\addcontentsline{toc}{section}{2. VERBATIM COPYING}
\end{center}

You may copy and distribute the Document in any medium, either
commercially or noncommercially, provided that this License, the
copyright notices, and the license notice saying this License applies
to the Document are reproduced in all copies, and that you add no other
conditions whatsoever to those of this License.  You may not use
technical measures to obstruct or control the reading or further
copying of the copies you make or distribute.  However, you may accept
compensation in exchange for copies.  If you distribute a large enough
number of copies you must also follow the conditions in section~3.

You may also lend copies, under the same conditions stated above, and
you may publicly display copies.


\begin{center}
{\Large\bf 3. COPYING IN QUANTITY\par}
\phantomsection
\addcontentsline{toc}{section}{3. COPYING IN QUANTITY}
\end{center}


If you publish printed copies (or copies in media that commonly have
printed covers) of the Document, numbering more than 100, and the
Document's license notice requires Cover Texts, you must enclose the
copies in covers that carry, clearly and legibly, all these Cover
Texts: Front-Cover Texts on the front cover, and Back-Cover Texts on
the back cover.  Both covers must also clearly and legibly identify
you as the publisher of these copies.  The front cover must present
the full title with all words of the title equally prominent and
visible.  You may add other material on the covers in addition.
Copying with changes limited to the covers, as long as they preserve
the title of the Document and satisfy these conditions, can be treated
as verbatim copying in other respects.

If the required texts for either cover are too voluminous to fit
legibly, you should put the first ones listed (as many as fit
reasonably) on the actual cover, and continue the rest onto adjacent
pages.

If you publish or distribute Opaque copies of the Document numbering
more than 100, you must either include a machine-readable Transparent
copy along with each Opaque copy, or state in or with each Opaque copy
a computer-network location from which the general network-using
public has access to download using public-standard network protocols
a complete Transparent copy of the Document, free of added material.
If you use the latter option, you must take reasonably prudent steps,
when you begin distribution of Opaque copies in quantity, to ensure
that this Transparent copy will remain thus accessible at the stated
location until at least one year after the last time you distribute an
Opaque copy (directly or through your agents or retailers) of that
edition to the public.

It is requested, but not required, that you contact the authors of the
Document well before redistributing any large number of copies, to give
them a chance to provide you with an updated version of the Document.


\begin{center}
{\Large\bf 4. MODIFICATIONS\par}
\phantomsection
\addcontentsline{toc}{section}{4. MODIFICATIONS}
\end{center}

You may copy and distribute a Modified Version of the Document under
the conditions of sections 2 and 3 above, provided that you release
the Modified Version under precisely this License, with the Modified
Version filling the role of the Document, thus licensing distribution
and modification of the Modified Version to whoever possesses a copy
of it.  In addition, you must do these things in the Modified Version:

\begin{itemize}
\item[A.] 
   Use in the Title Page (and on the covers, if any) a title distinct
   from that of the Document, and from those of previous versions
   (which should, if there were any, be listed in the History section
   of the Document).  You may use the same title as a previous version
   if the original publisher of that version gives permission.
   
\item[B.]
   List on the Title Page, as authors, one or more persons or entities
   responsible for authorship of the modifications in the Modified
   Version, together with at least five of the principal authors of the
   Document (all of its principal authors, if it has fewer than five),
   unless they release you from this requirement.
   
\item[C.]
   State on the Title page the name of the publisher of the
   Modified Version, as the publisher.
   
\item[D.]
   Preserve all the copyright notices of the Document.
   
\item[E.]
   Add an appropriate copyright notice for your modifications
   adjacent to the other copyright notices.
   
\item[F.]
   Include, immediately after the copyright notices, a license notice
   giving the public permission to use the Modified Version under the
   terms of this License, in the form shown in the Addendum below.
   
\item[G.]
   Preserve in that license notice the full lists of Invariant Sections
   and required Cover Texts given in the Document's license notice.
   
\item[H.]
   Include an unaltered copy of this License.
   
\item[I.]
   Preserve the section Entitled ``History'', Preserve its Title, and add
   to it an item stating at least the title, year, new authors, and
   publisher of the Modified Version as given on the Title Page.  If
   there is no section Entitled ``History'' in the Document, create one
   stating the title, year, authors, and publisher of the Document as
   given on its Title Page, then add an item describing the Modified
   Version as stated in the previous sentence.
   
\item[J.]
   Preserve the network location, if any, given in the Document for
   public access to a Transparent copy of the Document, and likewise
   the network locations given in the Document for previous versions
   it was based on.  These may be placed in the ``History'' section.
   You may omit a network location for a work that was published at
   least four years before the Document itself, or if the original
   publisher of the version it refers to gives permission.
   
\item[K.]
   For any section Entitled ``Acknowledgements'' or ``Dedications'',
   Preserve the Title of the section, and preserve in the section all
   the substance and tone of each of the contributor acknowledgements
   and/or dedications given therein.
   
\item[L.]
   Preserve all the Invariant Sections of the Document,
   unaltered in their text and in their titles.  Section numbers
   or the equivalent are not considered part of the section titles.
   
\item[M.]
   Delete any section Entitled ``Endorsements''.  Such a section
   may not be included in the Modified Version.
   
\item[N.]
   Do not retitle any existing section to be Entitled ``Endorsements''
   or to conflict in title with any Invariant Section.
   
\item[O.]
   Preserve any Warranty Disclaimers.
\end{itemize}

If the Modified Version includes new front-matter sections or
appendices that qualify as Secondary Sections and contain no material
copied from the Document, you may at your option designate some or all
of these sections as invariant.  To do this, add their titles to the
list of Invariant Sections in the Modified Version's license notice.
These titles must be distinct from any other section titles.

You may add a section Entitled ``Endorsements'', provided it contains
nothing but endorsements of your Modified Version by various
parties---for example, statements of peer review or that the text has
been approved by an organization as the authoritative definition of a
standard.

You may add a passage of up to five words as a Front-Cover Text, and a
passage of up to 25 words as a Back-Cover Text, to the end of the list
of Cover Texts in the Modified Version.  Only one passage of
Front-Cover Text and one of Back-Cover Text may be added by (or
through arrangements made by) any one entity.  If the Document already
includes a cover text for the same cover, previously added by you or
by arrangement made by the same entity you are acting on behalf of,
you may not add another; but you may replace the old one, on explicit
permission from the previous publisher that added the old one.

The author(s) and publisher(s) of the Document do not by this License
give permission to use their names for publicity for or to assert or
imply endorsement of any Modified Version.


\begin{center}
{\Large\bf 5. COMBINING DOCUMENTS\par}
\phantomsection
\addcontentsline{toc}{section}{5. COMBINING DOCUMENTS}
\end{center}


You may combine the Document with other documents released under this
License, under the terms defined in section~4 above for modified
versions, provided that you include in the combination all of the
Invariant Sections of all of the original documents, unmodified, and
list them all as Invariant Sections of your combined work in its
license notice, and that you preserve all their Warranty Disclaimers.

The combined work need only contain one copy of this License, and
multiple identical Invariant Sections may be replaced with a single
copy.  If there are multiple Invariant Sections with the same name but
different contents, make the title of each such section unique by
adding at the end of it, in parentheses, the name of the original
author or publisher of that section if known, or else a unique number.
Make the same adjustment to the section titles in the list of
Invariant Sections in the license notice of the combined work.

In the combination, you must combine any sections Entitled ``History''
in the various original documents, forming one section Entitled
``History''; likewise combine any sections Entitled ``Acknowledgements'',
and any sections Entitled ``Dedications''.  You must delete all sections
Entitled ``Endorsements''.

\begin{center}
{\Large\bf 6. COLLECTIONS OF DOCUMENTS\par}
\phantomsection
\addcontentsline{toc}{section}{6. COLLECTIONS OF DOCUMENTS}
\end{center}

You may make a collection consisting of the Document and other documents
released under this License, and replace the individual copies of this
License in the various documents with a single copy that is included in
the collection, provided that you follow the rules of this License for
verbatim copying of each of the documents in all other respects.

You may extract a single document from such a collection, and distribute
it individually under this License, provided you insert a copy of this
License into the extracted document, and follow this License in all
other respects regarding verbatim copying of that document.


\begin{center}
{\Large\bf 7. AGGREGATION WITH INDEPENDENT WORKS\par}
\phantomsection
\addcontentsline{toc}{section}{7. AGGREGATION WITH INDEPENDENT WORKS}
\end{center}


A compilation of the Document or its derivatives with other separate
and independent documents or works, in or on a volume of a storage or
distribution medium, is called an ``aggregate'' if the copyright
resulting from the compilation is not used to limit the legal rights
of the compilation's users beyond what the individual works permit.
When the Document is included in an aggregate, this License does not
apply to the other works in the aggregate which are not themselves
derivative works of the Document.

If the Cover Text requirement of section~3 is applicable to these
copies of the Document, then if the Document is less than one half of
the entire aggregate, the Document's Cover Texts may be placed on
covers that bracket the Document within the aggregate, or the
electronic equivalent of covers if the Document is in electronic form.
Otherwise they must appear on printed covers that bracket the whole
aggregate.


\begin{center}
{\Large\bf 8. TRANSLATION\par}
\phantomsection
\addcontentsline{toc}{section}{8. TRANSLATION}
\end{center}


Translation is considered a kind of modification, so you may
distribute translations of the Document under the terms of section~4.
Replacing Invariant Sections with translations requires special
permission from their copyright holders, but you may include
translations of some or all Invariant Sections in addition to the
original versions of these Invariant Sections.  You may include a
translation of this License, and all the license notices in the
Document, and any Warranty Disclaimers, provided that you also include
the original English version of this License and the original versions
of those notices and disclaimers.  In case of a disagreement between
the translation and the original version of this License or a notice
or disclaimer, the original version will prevail.

If a section in the Document is Entitled ``Acknowledgements'',
``Dedications'', or ``History'', the requirement (section~4) to Preserve
its Title (section~1) will typically require changing the actual
title.


\begin{center}
{\Large\bf 9. TERMINATION\par}
\phantomsection
\addcontentsline{toc}{section}{9. TERMINATION}
\end{center}


You may not copy, modify, sublicense, or distribute the Document
except as expressly provided under this License.  Any attempt
otherwise to copy, modify, sublicense, or distribute it is void, and
will automatically terminate your rights under this License.

However, if you cease all violation of this License, then your license
from a particular copyright holder is reinstated (a) provisionally,
unless and until the copyright holder explicitly and finally
terminates your license, and (b) permanently, if the copyright holder
fails to notify you of the violation by some reasonable means prior to
60 days after the cessation.

Moreover, your license from a particular copyright holder is
reinstated permanently if the copyright holder notifies you of the
violation by some reasonable means, this is the first time you have
received notice of violation of this License (for any work) from that
copyright holder, and you cure the violation prior to 30 days after
your receipt of the notice.

Termination of your rights under this section does not terminate the
licenses of parties who have received copies or rights from you under
this License.  If your rights have been terminated and not permanently
reinstated, receipt of a copy of some or all of the same material does
not give you any rights to use it.


\begin{center}
{\Large\bf 10. FUTURE REVISIONS OF THIS LICENSE\par}
\phantomsection
\addcontentsline{toc}{section}{10. FUTURE REVISIONS OF THIS LICENSE}
\end{center}


The Free Software Foundation may publish new, revised versions
of the GNU Free Documentation License from time to time.  Such new
versions will be similar in spirit to the present version, but may
differ in detail to address new problems or concerns.  See
http://www.gnu.org/copyleft/.

Each version of the License is given a distinguishing version number.
If the Document specifies that a particular numbered version of this
License ``or any later version'' applies to it, you have the option of
following the terms and conditions either of that specified version or
of any later version that has been published (not as a draft) by the
Free Software Foundation.  If the Document does not specify a version
number of this License, you may choose any version ever published (not
as a draft) by the Free Software Foundation.  If the Document
specifies that a proxy can decide which future versions of this
License can be used, that proxy's public statement of acceptance of a
version permanently authorizes you to choose that version for the
Document.


\begin{center}
{\Large\bf 11. RELICENSING\par}
\phantomsection
\addcontentsline{toc}{section}{11. RELICENSING}
\end{center}


``Massive Multiauthor Collaboration Site'' (or ``MMC Site'') means any
World Wide Web server that publishes copyrightable works and also
provides prominent facilities for anybody to edit those works.  A
public wiki that anybody can edit is an example of such a server.  A
``Massive Multiauthor Collaboration'' (or ``MMC'') contained in the
site means any set of copyrightable works thus published on the MMC
site.

``CC-BY-SA'' means the Creative Commons Attribution-Share Alike 3.0
license published by Creative Commons Corporation, a not-for-profit
corporation with a principal place of business in San Francisco,
California, as well as future copyleft versions of that license
published by that same organization.

``Incorporate'' means to publish or republish a Document, in whole or
in part, as part of another Document.

An MMC is ``eligible for relicensing'' if it is licensed under this
License, and if all works that were first published under this License
somewhere other than this MMC, and subsequently incorporated in whole
or in part into the MMC, (1) had no cover texts or invariant sections,
and (2) were thus incorporated prior to November 1, 2008.

The operator of an MMC Site may republish an MMC contained in the site
under CC-BY-SA on the same site at any time before August 1, 2009,
provided the MMC is eligible for relicensing.


\begin{center}
{\Large\bf ADDENDUM: How to use this License for your documents\par}
\phantomsection
\addcontentsline{toc}{section}{ADDENDUM: How to use this License for your documents}
\end{center}

To use this License in a document you have written, include a copy of
the License in the document and put the following copyright and
license notices just after the title page:

\bigskip
\begin{quote}
    Copyright \copyright{}  YEAR  YOUR NAME.
    Permission is granted to copy, distribute and/or modify this document
    under the terms of the GNU Free Documentation License, Version 1.3
    or any later version published by the Free Software Foundation;
    with no Invariant Sections, no Front-Cover Texts, and no Back-Cover Texts.
    A copy of the license is included in the section entitled ``GNU
    Free Documentation License''.
\end{quote}
\bigskip
    
If you have Invariant Sections, Front-Cover Texts and Back-Cover Texts,
replace the ``with \dots\ Texts.'' line with this:

\bigskip
\begin{quote}
    with the Invariant Sections being LIST THEIR TITLES, with the
    Front-Cover Texts being LIST, and with the Back-Cover Texts being LIST.
\end{quote}
\bigskip
    
If you have Invariant Sections without Cover Texts, or some other
combination of the three, merge those two alternatives to suit the
situation.

If your document contains nontrivial examples of program code, we
recommend releasing these examples in parallel under your choice of
free software license, such as the GNU General Public License,
to permit their use in free software.



\printindex

\listoflistings

\end{document}

