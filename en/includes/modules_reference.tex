\chapter{Stock Modules Reference}

Augeas comes with an extensive collection of lenses to parse common configuration files. This chapter will cover the most common of these lenses and provide a reference and examples for them.

This list is not complete. The complete documentation of the modules can be obtained by building it from the Augeas source\footnote{See \emph{installing Augeas} on page \pageref{sec:installing_augeas}}.

\section{The Access Module}

Description of the module

\section{The Aliases Module}

Description of the module

\section{The Approx Module}

Description of the module

\section{The AptPreferences Module}

Description of the module

\section{The Aptsources Module}

Description of the module

\section{The Cron Module}

Description of the module

\section{The Crypttab Module}

Description of the module

\section{The Dhclient Module}

Description of the module

\section{The Dhcpd Module}

Description of the module

\section{The Dnsmasq Module}

Description of the module

\section{The Ethers Module}

Description of the module

\section{The Exports Module}

Description of the module

\section{The Fstab Module}

The \verb!Fstab! modules parses \verb!/etc/fstab! and \verb!/etc/mtab!.

\subsection{Anatomy}

The root nodes are either \verb!#comment! nodes or \verb!seq! nodes.

Each of the \verb!seq! nodes correspond to a line in the file, and has the following nodes:

\begin{itemize}
\item
  A \verb!spec! node, the block special device or remote filesystem to be mounted ;
\item
  A \verb!file! node, the mount point for the file system ;
\item
  A \verb!vfstype! node, the type of filesystem ;
\item
  One or more \verb!opt! nodes, the mount options associated with the filesystem ;
\item
  An optional \verb!dump! node, the parameter for the \verb!dump(8)! command ;
\item
  An optional \verb!passno! node, the parameter for the \verb!fsck(8)! command.
\end{itemize}
See \verb!man fstab! for more information on these fields.

\subsection{Sample tree}

\begin{verbatim}
/files/etc/fstab
/files/etc/fstab/#comment[1] = "/etc/fstab: static file system information."
/files/etc/fstab/#comment[2] = "Use 'blkid -o value -s UUID' to print the universally unique identifier"
/files/etc/fstab/#comment[3] = "for a device; this may be used with UUID= as a more robust way to name"
/files/etc/fstab/#comment[4] = "devices that works even if disks are added and removed. See fstab(5)."
/files/etc/fstab/#comment[5] = "<file system> <mount point>   <type>  <options>       <dump>  <pass>"
/files/etc/fstab/1
/files/etc/fstab/1/spec = "proc"
/files/etc/fstab/1/file = "/proc"
/files/etc/fstab/1/vfstype = "proc"
/files/etc/fstab/1/opt[1] = "nodev"
/files/etc/fstab/1/opt[2] = "noexec"
/files/etc/fstab/1/opt[3] = "nosuid"
/files/etc/fstab/1/dump = "0"
/files/etc/fstab/1/passno = "0"
/files/etc/fstab/#comment[6] = "/ was on /dev/sda1 during installation"
/files/etc/fstab/2
/files/etc/fstab/2/spec = "UUID=4cbb4f80-45f9-4e46-a076-8ec1124f4835"
/files/etc/fstab/2/file = "/"
/files/etc/fstab/2/vfstype = "ext3"
/files/etc/fstab/2/opt = "errors"
/files/etc/fstab/2/opt/value = "remount-ro"
/files/etc/fstab/2/dump = "0"
/files/etc/fstab/2/passno = "1"
/files/etc/fstab/3
/files/etc/fstab/3/spec = "/dev/sda2"
/files/etc/fstab/3/file = "none"
/files/etc/fstab/3/vfstype = "swap"
/files/etc/fstab/3/opt = "rw"
/files/etc/fstab/3/dump = "0"
/files/etc/fstab/3/passno = "0"
\end{verbatim}
\subsection{Examples of path expressions}

\begin{verbatim}
match '/files/etc/fstab/*[label() != "#comment"]'
\end{verbatim}
matches all lines except comment lines.

\begin{verbatim}
match '/files/etc/fstab/*[file = "/"]/spec'
\end{verbatim}
matches the \verb!spec! node for the filesystem mounted on ``/''.

\begin{verbatim}
rm '/files/etc/fstab/*/opt[. = "nosuid"]'
\end{verbatim}
removes all \verb!opt! nodes whose value is \verb!nosuid!.

\section{The Group Module}

Description of the module

\section{The Grub Module}

Description of the module

\section{The Hosts Module}

Description of the module

\section{The Httpd Module}

Description of the module

\section{The Inetd Module}

Description of the module

\section{The Inittab Module}

Description of the module

\section{The Interfaces Module}

Description of the module

\section{The Iptables Module}

Description of the module

\section{The Json Module}

Description of the module

\section{The Limits Module}

Description of the module

\section{The Login\_defs Module}

Description of the module

\section{The Logrotate Module}

Description of the module

\section{The Modprobe Module}

Description of the module

\section{The Modules\_conf Module}

Description of the module

\section{The MySQL Module}

Description of the module

\section{The Nsswitch Module}

Description of the module

\section{The Ntp Module}

Description of the module

\section{The Pam Module}

Description of the module

\section{The Passwd Module}

Description of the module

\section{The Pg\_Hba Module}

Description of the module

\section{The PHP Module}

Description of the module

\section{The Postfix\_Master Module}

Description of the module

\section{The Puppet Module}

Description of the module

\section{The Resolv Module}

Description of the module

\section{The Services Module}

Description of the module

\section{The Shells Module}

Description of the module

\section{The Sshd Module}

Description of the module

\section{The Sudoers Module}

Description of the module

\section{The Sysconfig Module}

Description of the module

\section{The Sysctl Module}

Description of the module

\section{The Syslog Module}

Description of the module

\section{The Xinetd Module}

Description of the module

\section{The Xml Module}

Description of the module

\section{The Yum Module}

Description of the module


